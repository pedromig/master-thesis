\begin{frame}{Optimization}

  \begin{block}{Optimization Problem}
    An optimization problem is a tuple $(\mathcal{S}, f)$, where
    $\mathcal{S}$ is a set containing all feasible solutions, and $f$ is an
    objective (cost) function, with a mapping such that:
    \begin{equation*}
      f \colon \mathcal{S} \longrightarrow \mathbb{R}
    \end{equation*}
  \end{block}

  \begin{block}{Combinatorial Optimization Problem}
    A combinatorial optimization problem is an optimization problem where the
    set $\mathcal{S}$ of feasible solutions is finite.
  \end{block}

  In this work we assume without loss of generality a maximizing objective function.
\end{frame}

\begin{frame}{Ground Set}

  \begin{block}{Ground Set}
    The ground set of a CO problem is a finite set of
    components $\mathcal{G} = \{c_1, c_2, \ldots, c_k\}$, such that every solution to the
    problem, feasible or not, can be defined as a subset of $\mathcal{G}$.
  \end{block}

  \begin{block}{Empty Solution}
    A solution $s \in 2^\mathcal{G}$, where $2^\mathcal{G}$ denotes the powerset of $\mathcal{G}$,
    is said to be an empty solution if $s = \emptyset$.
  \end{block}

  \begin{block}{Partial Solution}
    A solution $s \in 2^\mathcal{G}$ is said to be a
    partial solution if there is a feasible solution $s' \in \mathcal{S}$ such
    that $s' \supseteq s$.
  \end{block}

  \begin{block}{Complete Solution}
    A feasible solution $s \in \mathcal{S}$ is said
    to be a complete solution if there is no feasible solution $s' \in
      \mathcal{S}$ such that $s' \supset s$.
  \end{block}
\end{frame}

\begin{frame}{Bounds}
  The usage of~\alert{bounds}, particularly the~\emph{upper bound} in a maximization setting,
  is beneficial as it aids in:

  \begin{itemize}
    \item The evaluation of infeasible solutions.
    \item The measurement of a candidate solution's potential.
  \end{itemize}

  \begin{block}{Upper Bound}
    An upper bound of a (partial) solution $s \in 2^\mathcal{G}$ is any numeric
    value given by a function $\Phi_\text{ub}\colon 2^{\mathcal{G}} \rightarrow
      \mathbb{R} $ such that:
    \begin{equation*}
      \forall s' \in \mathcal{S} \land s' \supseteq s \colon f(s') \le \Phi_\text{ub}(s)
    \end{equation*}
  \end{block}
\end{frame}

\begin{frame}[fragile]{Constructive Search}
  \alert{Constructive Search} is a procedure for optimization that operates as follows:

  \begin{itemize}
    \item Initiate the process with an empty or partial solution
    \item Add a component from the ground set to the solution.
    \item Repeat the process until no more (feasible) components are available.
  \end{itemize}

  \begin{center}
    \scalebox{0.75}{%  
      \begin{algorithm}[H]
        \KwIn{Ground Set ($\mathcal{G}$)}
\KwOut{Solution ($s$)}

$s$ $\gets$ $\emptyset$\;
$C$ $\gets$ $\{$  $c\in \mathcal{G} \setminus s$ $|$ $s \cup \{c\}$ \texttt{is feasible} $\}$ \;
\While{$C \neq \emptyset$}{
  $c$ $\gets$ \texttt{SelectComponent($C$)}\;
  $s$ $\gets$ $s \cup \{c\}$\;
  $C$ $\gets$ $\{$  $c\in \mathcal{G} \setminus s$ $|$ $s \cup \{c\}$ \texttt{is feasible} $\}$ \;
}
\Return{$s$}

        \caption{Constructive Search Procedure}
      \end{algorithm}
    }
  \end{center}
\end{frame}


\begin{frame}{Local Search}
  \alert{Local Search} is a procedure for optimization that operates as follows:

  \begin{itemize}
    \item Start with a feasible solution to the problem.
    \item Apply local moves and perturbations to the solution, thus exploring candidate solutions in the neighborhood.
    \item Repeat the process until the solution cannot be further improved.
  \end{itemize}

  \begin{center}
    \scalebox{0.75}{%  
      \begin{algorithm}[H]
        % Local Search Procedure (algorithm2e pseudocode)

\KwIn{Solution ($s'$)}
\KwOut{Solution ($s$)}

\SetKwFunction{LocalMoves}{\texttt{LocalMoves}}
\SetKwFunction{SelectLocalMove}{\texttt{SelectLocalMove}}
\SetKwFunction{Step}{\texttt{Step}}
\SetKwFunction{Perturb}{\texttt{Perturb}}

$s$ $\gets$ $s'$\;
$\mathcal{M}$ $\gets$ \LocalMoves{$s$}\;
\While{$\mathcal{M} \neq \emptyset$ }{
  $m$ $\gets$ \SelectLocalMove{$\mathcal{M}$}\;
  $s$ $\gets$ \Step{$s, m$}\;
  $s$ $\gets$ \Perturb{$s$} \Comment*[f]{Optional}\;
  $\mathcal{M}$ $\gets$ \LocalMoves{$s$}
}
\Return{ $s$ }
        \caption{Local Search Procedure}
      \end{algorithm}
    }
  \end{center}
\end{frame}

\begin{frame}{Meta-Heuristics}
  The following~\alert{meta-heuristics}, were analyzed and implemented in the context
  of this work.
  \begin{itemize}
    \item \emph{Beam Search}
    \item \emph{Iterated Greedy}
    \item \emph{Greedy Randomized Adaptive Search Procedure}
    \item \emph{Ant Colony Optimization}
    \item \emph{Hill-Climbing}
    \item \emph{Iterated Local Search}
    \item \emph{Simulated Annealing}
    \item \emph{Tabu Search}
  \end{itemize}

\end{frame}