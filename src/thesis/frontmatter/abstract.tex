\begin{abstract}

  Combinatorial Optimization problems are ubiquitous in real-world scenarios. To
  solve these, there are a wide range of methods described in the literature from
  which we highlight exact and meta-heuristic methods. Exact methods can find
  optimal solutions. However, they are often infeasible in practice due to the
  NP-Hard nature of most combinatorial optimization problems. On the other hand,
  meta-heuristics often cannot provably find optimal solutions, but can find
  solutions that are of "good" quality, which motivates the growing
  interest in their use for combinatorial problems.

  There has been growing interest in the development of a general-purpose
  framework for the development of black-box meta-heuristic methods that
  separates problem-specific from approach-specific details. In this work, we
  build upon this idea and expand on previous work that has looked into the
  development of a framework for constructive and local search meta-heuristic
  approaches. In particular we give a general framework and corresponding Python
  implementation that encompasses both search approaches, and implement several
  common algorithms under this framework.

  In addition, there is a growing interest in the community in developing a suite
  of benchmark problems for accessing the quality of meta-heuristics strategies.
  The Google hash code problems, being combinatorial problems in nature and
  modelled after real-world scenarios pose themselves as interesting candidates
  for this purpose. In this work, we analyze all Google Hash Code problems, and
  implement several models for two of the problems under our general framework to
  show that it allows for the development of models that give very competitive
  results in practice.

\end{abstract}

\begin{abstract}[Resumo]

  Os problemas de Otimização Combinatória são ubíquos em cenários da vida real.
  Para resolver estes problemas, existe uma grande variedade de métodos descritos
  na literatura, nos quais se destacam métodos exatos e métodos meta-heurísticos.
  Embora os métodos exatos consigam encontrar soluções ótimas, estes, na prática,
  são frequentemente inviáveis devido à natureza NP-Difícil da maioria dos
  problemas de otimização combinatória. Por outro lado, meta-heurísticas são
  tipicamente incapazes de encontrar soluções ótimas, no entanto, conseguem
  encontrar soluções de~\textquote{boa} qualidade, o que motiva o crescente
  interesse na sua utilização em problemas combinatórios.

  Tem existido crescente interesse no desenvolvimento de uma plataforma para o
  desenvolvimento de métodos meta-heurísticos de caixa preta que separam
  detalhes específicos do problema de detalhes específicos da abordagem. Neste
  trabalho, construímos sobre essa ideia e expandimos trabalhos anteriores que
  investigaram o desenvolvimento de uma plataforma para abordagens
  meta-heurísticas de procura construtiva e local. Em particular, desenvolvemos
  uma plataforma em Python que reúne ambas as abordagens de procura e
  implementamos também vários algoritmos que seguem esta abordagem.

  Para além disso, há também um crescente interesse na comunidade em desenvolver
  um conjunto de problemas de referência para avaliar a qualidade das
  estratégias meta-heurísticas. Os problemas do Google Hash Code, por serem
  problemas combinatórios e baseados em cenários do mundo real, apresentam-se
  como candidatos interessantes para esta análise. Neste trabalho, analisamos todos
  os problemas do Google Hash Code e implementamos vários modelos para dois
  desses problemas, demonstrando que ele permite o desenvolvimento de modelos
  que proporcionam resultados muito competitivos na prática.

\end{abstract}