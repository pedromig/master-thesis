\documentclass[oneside]{thesis}

\usepackage{./latex-thesis-cover/uccover}

\usepackage{tikz,pgfplots,pgfplotstable}

\usepackage{lipsum}

% Color Definitions
\definecolor{cb_blue}{RGB}{0, 92, 169}     % Blue
\definecolor{cb_orange}{RGB}{255, 140, 0}  % Orange
\definecolor{cb_green}{RGB}{0, 158, 115}   % Green
\definecolor{cb_purple}{RGB}{102, 51, 153} % Purple
\definecolor{cb_pink}{RGB}{255, 187, 120}  % Pink
\definecolor{cb_gray}{RGB}{128, 128, 128}  % Gray



% PDF Document Metadata
\hypersetup{
 pdftitle = {Principled Modelling Of The Google Hash Code Problems For Meta-Heuristics},
 pdfauthor = {Pedro Rodrigues, Alexandre Jesus, Carlos Fonseca},
 pdfsubject = {Master Thesis},
 pdfkeywords = {Modelling, Meta-Heuristics, Combinatorial Optimization, Constructive Search, Local Search, Intelligent Systems, Software Engineering},
 pdfproducer = {LuaLaTeX},
 pdfcreator = {Pedro Rodrigues},
}

% Bibliography
\addbibresource{../references.bib}

% Acronyms
\newacronym{cisuc}{CISUC}{Centre for Informatics and Systems of the University of Coimbra}
\newacronym{ac}{AC}{Adaptive Computation}
\newacronym{algo}{ALGO}{Algorithms and Optimization Laboratory}
\newacronym{fct}{FCT}{Foundation for Science and Technology}

\newacronym{combinatorial-optimization}{CO}{Combinatorial Optimization}
\newacronym{global-optimization}{GO}{Global Optimization}
\newacronym{local-optimization}{LO}{Local Optimization}
\newacronym{black-box-optimization}{BBO}{Black Box Optimization}
\newacronym{glass-box-optimization}{GBO}{Glass Box Optimization}

\newacronym{knapsack-problem}{KP}{Knapsack Problem}
\newacronym{travelling-salesman-problem}{TSP}{Travelling Salesman Problem}

\makeglossaries

\begin{document}

\frontmatter

% Institutional Thesis Cover
\ifthenelse{\boolean{@twoside}}{
  \makecover{Principled Modeling Of The Google \\ Hash Code Problems For \\ Meta-Heuristics\\}{}{}{Pedro Miguel Duque Rodrigues}{September
    2023}{Dissertation in the context of the Master in Informatics Engineering,
    Specialization in Intelligent Systems, advised by Professor Alexandre D. Jesus
    and Professor Carlos M. Fonseca, and presented to the Faculty of Sciences and
    Technology / Department of Informatics Engineering.}
  \blankpage
}{}
\relax 
\providecommand\hyper@newdestlabel[2]{}
\@setckpt{frontmatter/cover}{
\setcounter{page}{2}
\setcounter{equation}{0}
\setcounter{enumi}{0}
\setcounter{enumii}{0}
\setcounter{enumiii}{0}
\setcounter{enumiv}{0}
\setcounter{footnote}{0}
\setcounter{mpfootnote}{0}
\setcounter{part}{0}
\setcounter{chapter}{0}
\setcounter{section}{0}
\setcounter{subsection}{0}
\setcounter{subsubsection}{0}
\setcounter{paragraph}{0}
\setcounter{subparagraph}{0}
\setcounter{figure}{0}
\setcounter{table}{0}
\setcounter{parentequation}{0}
\setcounter{Item}{0}
\setcounter{Hfootnote}{0}
\setcounter{bookmark@seq@number}{0}
\setcounter{AM@survey}{0}
\setcounter{section@level}{0}
}

\begin{funding}
  The work presented in this thesis was carried out in the Algorithms and
  Optimization Laboratory of the Adaptive Computation group of the Centre for
  Informatics and Systems of the University of Coimbra and financially supported
  by the Foundation for Science and Technology under a scholarship with
  reference UIDP/00326/2020.

  \copyright{}~2023 Pedro Rodrigues
\end{funding}
\begin{abstract}

  Combinatorial Optimization problems are ubiquitous in real-world scenarios. To
  solve these, there are a wide range of methods described in the literature from
  which we highlight exact and meta-heuristic methods. Exact methods can find
  optimal solutions. However, they are often infeasible in practice due to the
  NP-Hard nature of most combinatorial optimization problems. On the other hand,
  meta-heuristics often cannot provably find optimal solutions, but can find
  solutions that are of "good" quality, which motivates the growing
  interest in their use for combinatorial problems.

  There has been growing interest in the development of a general-purpose
  framework for the development of black-box meta-heuristic methods that
  separates problem-specific from approach-specific details. In this work, we
  build upon this idea and expand on previous work that has looked into the
  development of a framework for constructive and local search meta-heuristic
  approaches. In particular we give a general framework and corresponding Python
  implementation that encompasses both search approaches, and implement several
  common algorithms under this framework.

  In addition, there is a growing interest in the community in developing a suite
  of benchmark problems for accessing the quality of meta-heuristics strategies.
  The Google hash code problems, being combinatorial problems in nature and
  modelled after real-world scenarios pose themselves as interesting candidates
  for this purpose. In this work, we analyze all Google Hash Code problems, and
  implement several models for two of the problems under our general framework to
  show that it allows for the development of models that give very competitive
  results in practice.

\end{abstract}

\begin{abstract}[Resumo]

  Os problemas de Otimização Combinatória são ubíquos em cenários da vida real.
  Para resolver estes problemas, existe uma grande variedade de métodos descritos
  na literatura, nos quais se destacam métodos exatos e métodos meta-heurísticos.
  Embora os métodos exatos consigam encontrar soluções ótimas, estes, na prática,
  são frequentemente inviáveis devido à natureza NP-Difícil da maioria dos
  problemas de otimização combinatória. Por outro lado, meta-heurísticas são
  tipicamente incapazes de encontrar soluções ótimas, no entanto, conseguem
  encontrar soluções de~\textquote{boa} qualidade, o que motiva o crescente
  interesse na sua utilização em problemas combinatórios.

  Tem existido crescente interesse no desenvolvimento de uma plataforma para o
  desenvolvimento de métodos meta-heurísticos de caixa preta que separam
  detalhes específicos do problema de detalhes específicos da abordagem. Neste
  trabalho, construímos sobre essa ideia e expandimos trabalhos anteriores que
  investigaram o desenvolvimento de uma plataforma para abordagens
  meta-heurísticas de procura construtiva e local. Em particular, desenvolvemos
  uma plataforma em Python que reúne ambas as abordagens de procura e
  implementamos também vários algoritmos que seguem esta abordagem.

  Para além disso, há também um crescente interesse na comunidade em desenvolver
  um conjunto de problemas de referência para avaliar a qualidade das
  estratégias meta-heurísticas. Os problemas do Google Hash Code, por serem
  problemas combinatórios e baseados em cenários do mundo real, apresentam-se
  como candidatos interessantes para esta análise. Neste trabalho, analisamos
  todos os problemas do Google Hash Code e implementamos vários modelos para
  dois desses problemas, demonstrando que a nossa framework permite o
  desenvolvimento de modelos que proporcionam resultados muito competitivos na
  prática.

\end{abstract}
\begin{acknowledgments}

  As I conclude my thesis, I am profoundly grateful for the invaluable support and
  guidance I have received from numerous individuals who have made my academic
  journey possible.

  First and foremost, I owe my deepest thanks to Professor Alexandre Jesus and
  Carlos Fonseca, my supervisors, for their unwavering support, constant
  encouragement, and unwavering inspiration throughout my academic journey. They
  have consistently made time to listen to my challenges and provided assistance
  whenever needed. Most importantly, I deeply appreciate their mentorship and the
  friendship that has helped me develop both technically and personally.

  I extend my heartfelt gratitude to my~\textquote{university family}, particularly my
  \textquote{goddaughters}, who have shown me care and support when I needed it most. Each
  one of you has made a meaningful contribution to my days. I want to specifically
  thank Tatiana Almeida for her constant check-ins, Inês Marçal for sharing jokes
  and bringing laughter into my life, and Joana Antunes for her friendly waves in
  the DEI corridor whenever I filled my water bottle. Additionally, I want to
  express my gratitude to my~\textquote{grandson}, João, for being a constant presence
  during late-night visits to my dorm room, patiently listening to my vents and
  discussing our days. Your kindness and companionship have meant the world to me.

  I extend my thanks to all my friends who have been by my side throughout this
  journey. Although it's impossible to mention each one of you, please know that
  this work would not have been possible without your unwavering support. In
  particular, I would like to express my gratitude to Samuel Carinhas, my
  \textquote{wingman} in all our adventures, gaming nights, and other memorable moments. To Duarte
  Dias, Gabriel Fernandes, and Miguel Rabuge, your great discussions,
  collaborative project work, and cherished moments have made this journey so
  enjoyable. Mariana Loreto and André Silva, thank you for having faith in me,
  sometimes more than I did, and for always encouraging me to give my best.

  I must also extend special thanks to all the members of the Algorithms and
  Optimization Laboratory with whom I have shared countless moments and from whom
  I have learned invaluable lessons. I'm especially grateful to Luis Paquete for
  setting me on my research path and to Gonçalo Lopes for always keeping me
  company in the lab during lunch or over coffee. I also want to acknowledge the
  members of the Núcleo de Estudantes de Informática, who have made this year an
  unforgettable and pleasant experience.

  My gratitude also goes to my uncle (and godfather) Rui Duque and aunt Andreia
  Rosa for always being present, offering valuable advice, and helping me take a
  break when I needed it most.

  Last but certainly not least, none of this would have been possible without the
  unwavering support of my sister and my parents. They have always been there to
  provide for me, teach me, and support me in every way possible, and to them, I
  owe everything.

\end{acknowledgments}
\printglossary[type=\acronymtype]
\printbibliography[heading=bibintoc]

\mainmatter

\chapter{Conclusion}
\label{ch:conclusion}

\chquote{\textquote{We can only see a short distance ahead, but we can see plenty there that needs to be done.}}{Alan Turing}{}

In this thesis, our main objective was to address two complex combinatorial
optimization problems from the Google Hash Code competition using~\acrshort{meta-heuristic}
approaches.~Our approach was centered on developing conceptual models for these
problems and implementing them within a framework we designed. This enabled the
utilization of generic~\acrshort{meta-heuristic} solvers that were also developed to
conform to the framework's specification. Essentially, the framework acted as a
practical tool to bridge the gap between conceptual problem modeling and actual
problem-solving, while offering a platform for creating generic~\acrshort{meta-heuristic}
solvers that can tackle any problem in a black-box fashion.

Our efforts resulted in successful problem modeling and promising outcomes for
both of the problems we attempted to solve. Notably, all the problem models we
devised could be implemented through the framework, highlighting the
adaptability and effectiveness of our approach in tackling intricate problems.
Moreover, our framework allowed the development of all the~\acrshort{meta-heuristic}
strategies discussed in~\Cref{ch:background}, demonstrating that is indeed
possible to this. However, it's important to acknowledge that the framework we
constructed will benefit from further refinement, especially in terms of
accommodating different modeling aspects, such as describing models for other
types of meta-heuristics like evolutionary algorithms, which were not within the
scope of this work.

Overall, the experience of working with this framework has positive, primarily
due to its transparency in addressing essential modeling questions and its clear
separation of solvers and models, making a wide range of utilities readily
available. However, becoming proficient in the framework requires
training and conceptual familiarity, which will only come with practice.

\section{Future Work}

Possible research directions in future work are presented as follows:

\begin{description}
  \item[Problem Models] Expanding the repertoire of problem models, both for
    the existing Google Hash Code problems and for other challenges within the
    competition, holds significant importance. This broader range of models will
    enable a comprehensive assessment of how the framework performs when applied
    to problems with diverse characteristics. The models we've constructed so far
    represent just a small subset, and a more extensive selection will facilitate
    a more in-depth analysis of~\acrshort{meta-heuristic} behavior. This expansion is
    crucial, as having a variety of performant models will support extensive
    testing and refinement of solvers.

  \item[Meta-Heuristic Implementation] Further implementations of other
    meta-heuristic techniques, such as evolutionary algorithms, can be
    explored. The introduction of new meta-heuristics may necessitate updates
    and additional functionalities for the framework to accommodate these approaches effectively.

  \item[Experimental Evaluation of Meta-Heuristics] The results obtained from the
    developed models have not yet delved into the parametrization of meta-heuristic
    methods. This analysis is crucial for optimizing and fine-tuning these
    techniques to achieve the best possible performance in solving complex problems.
    Furthermore, future work should include a thorough examination and comparison
    of different  meta-heuristics.
\end{description}
\chapter{Conclusion}
\label{ch:conclusion}

\chquote{\textquote{We can only see a short distance ahead, but we can see plenty there that needs to be done.}}{Alan Turing}{}

In this thesis, our main objective was to address two complex combinatorial
optimization problems from the Google Hash Code competition using~\acrshort{meta-heuristic}
approaches.~Our approach was centered on developing conceptual models for these
problems and implementing them within a framework we designed. This enabled the
utilization of generic~\acrshort{meta-heuristic} solvers that were also developed to
conform to the framework's specification. Essentially, the framework acted as a
practical tool to bridge the gap between conceptual problem modeling and actual
problem-solving, while offering a platform for creating generic~\acrshort{meta-heuristic}
solvers that can tackle any problem in a black-box fashion.

Our efforts resulted in successful problem modeling and promising outcomes for
both of the problems we attempted to solve. Notably, all the problem models we
devised could be implemented through the framework, highlighting the
adaptability and effectiveness of our approach in tackling intricate problems.
Moreover, our framework allowed the development of all the~\acrshort{meta-heuristic}
strategies discussed in~\Cref{ch:background}, demonstrating that is indeed
possible to this. However, it's important to acknowledge that the framework we
constructed will benefit from further refinement, especially in terms of
accommodating different modeling aspects, such as describing models for other
types of meta-heuristics like evolutionary algorithms, which were not within the
scope of this work.

Overall, the experience of working with this framework has positive, primarily
due to its transparency in addressing essential modeling questions and its clear
separation of solvers and models, making a wide range of utilities readily
available. However, becoming proficient in the framework requires
training and conceptual familiarity, which will only come with practice.

\section{Future Work}

Possible research directions in future work are presented as follows:

\begin{description}
  \item[Problem Models] Expanding the repertoire of problem models, both for
    the existing Google Hash Code problems and for other challenges within the
    competition, holds significant importance. This broader range of models will
    enable a comprehensive assessment of how the framework performs when applied
    to problems with diverse characteristics. The models we've constructed so far
    represent just a small subset, and a more extensive selection will facilitate
    a more in-depth analysis of~\acrshort{meta-heuristic} behavior. This expansion is
    crucial, as having a variety of performant models will support extensive
    testing and refinement of solvers.

  \item[Meta-Heuristic Implementation] Further implementations of other
    meta-heuristic techniques, such as evolutionary algorithms, can be
    explored. The introduction of new meta-heuristics may necessitate updates
    and additional functionalities for the framework to accommodate these approaches effectively.

  \item[Experimental Evaluation of Meta-Heuristics] The results obtained from the
    developed models have not yet delved into the parametrization of meta-heuristic
    methods. This analysis is crucial for optimizing and fine-tuning these
    techniques to achieve the best possible performance in solving complex problems.
    Furthermore, future work should include a thorough examination and comparison
    of different  meta-heuristics.
\end{description}
\chapter{Conclusion}
\label{ch:conclusion}

\chquote{\textquote{We can only see a short distance ahead, but we can see plenty there that needs to be done.}}{Alan Turing}{}

In this thesis, our main objective was to address two complex combinatorial
optimization problems from the Google Hash Code competition using~\acrshort{meta-heuristic}
approaches.~Our approach was centered on developing conceptual models for these
problems and implementing them within a framework we designed. This enabled the
utilization of generic~\acrshort{meta-heuristic} solvers that were also developed to
conform to the framework's specification. Essentially, the framework acted as a
practical tool to bridge the gap between conceptual problem modeling and actual
problem-solving, while offering a platform for creating generic~\acrshort{meta-heuristic}
solvers that can tackle any problem in a black-box fashion.

Our efforts resulted in successful problem modeling and promising outcomes for
both of the problems we attempted to solve. Notably, all the problem models we
devised could be implemented through the framework, highlighting the
adaptability and effectiveness of our approach in tackling intricate problems.
Moreover, our framework allowed the development of all the~\acrshort{meta-heuristic}
strategies discussed in~\Cref{ch:background}, demonstrating that is indeed
possible to this. However, it's important to acknowledge that the framework we
constructed will benefit from further refinement, especially in terms of
accommodating different modeling aspects, such as describing models for other
types of meta-heuristics like evolutionary algorithms, which were not within the
scope of this work.

Overall, the experience of working with this framework has positive, primarily
due to its transparency in addressing essential modeling questions and its clear
separation of solvers and models, making a wide range of utilities readily
available. However, becoming proficient in the framework requires
training and conceptual familiarity, which will only come with practice.

\section{Future Work}

Possible research directions in future work are presented as follows:

\begin{description}
  \item[Problem Models] Expanding the repertoire of problem models, both for
    the existing Google Hash Code problems and for other challenges within the
    competition, holds significant importance. This broader range of models will
    enable a comprehensive assessment of how the framework performs when applied
    to problems with diverse characteristics. The models we've constructed so far
    represent just a small subset, and a more extensive selection will facilitate
    a more in-depth analysis of~\acrshort{meta-heuristic} behavior. This expansion is
    crucial, as having a variety of performant models will support extensive
    testing and refinement of solvers.

  \item[Meta-Heuristic Implementation] Further implementations of other
    meta-heuristic techniques, such as evolutionary algorithms, can be
    explored. The introduction of new meta-heuristics may necessitate updates
    and additional functionalities for the framework to accommodate these approaches effectively.

  \item[Experimental Evaluation of Meta-Heuristics] The results obtained from the
    developed models have not yet delved into the parametrization of meta-heuristic
    methods. This analysis is crucial for optimizing and fine-tuning these
    techniques to achieve the best possible performance in solving complex problems.
    Furthermore, future work should include a thorough examination and comparison
    of different  meta-heuristics.
\end{description}
\chapter{Conclusion}
\label{ch:conclusion}

\chquote{\textquote{We can only see a short distance ahead, but we can see plenty there that needs to be done.}}{Alan Turing}{}

In this thesis, our main objective was to address two complex combinatorial
optimization problems from the Google Hash Code competition using~\acrshort{meta-heuristic}
approaches.~Our approach was centered on developing conceptual models for these
problems and implementing them within a framework we designed. This enabled the
utilization of generic~\acrshort{meta-heuristic} solvers that were also developed to
conform to the framework's specification. Essentially, the framework acted as a
practical tool to bridge the gap between conceptual problem modeling and actual
problem-solving, while offering a platform for creating generic~\acrshort{meta-heuristic}
solvers that can tackle any problem in a black-box fashion.

Our efforts resulted in successful problem modeling and promising outcomes for
both of the problems we attempted to solve. Notably, all the problem models we
devised could be implemented through the framework, highlighting the
adaptability and effectiveness of our approach in tackling intricate problems.
Moreover, our framework allowed the development of all the~\acrshort{meta-heuristic}
strategies discussed in~\Cref{ch:background}, demonstrating that is indeed
possible to this. However, it's important to acknowledge that the framework we
constructed will benefit from further refinement, especially in terms of
accommodating different modeling aspects, such as describing models for other
types of meta-heuristics like evolutionary algorithms, which were not within the
scope of this work.

Overall, the experience of working with this framework has positive, primarily
due to its transparency in addressing essential modeling questions and its clear
separation of solvers and models, making a wide range of utilities readily
available. However, becoming proficient in the framework requires
training and conceptual familiarity, which will only come with practice.

\section{Future Work}

Possible research directions in future work are presented as follows:

\begin{description}
  \item[Problem Models] Expanding the repertoire of problem models, both for
    the existing Google Hash Code problems and for other challenges within the
    competition, holds significant importance. This broader range of models will
    enable a comprehensive assessment of how the framework performs when applied
    to problems with diverse characteristics. The models we've constructed so far
    represent just a small subset, and a more extensive selection will facilitate
    a more in-depth analysis of~\acrshort{meta-heuristic} behavior. This expansion is
    crucial, as having a variety of performant models will support extensive
    testing and refinement of solvers.

  \item[Meta-Heuristic Implementation] Further implementations of other
    meta-heuristic techniques, such as evolutionary algorithms, can be
    explored. The introduction of new meta-heuristics may necessitate updates
    and additional functionalities for the framework to accommodate these approaches effectively.

  \item[Experimental Evaluation of Meta-Heuristics] The results obtained from the
    developed models have not yet delved into the parametrization of meta-heuristic
    methods. This analysis is crucial for optimizing and fine-tuning these
    techniques to achieve the best possible performance in solving complex problems.
    Furthermore, future work should include a thorough examination and comparison
    of different  meta-heuristics.
\end{description}
\chapter{Conclusion}
\label{ch:conclusion}

\chquote{\textquote{We can only see a short distance ahead, but we can see plenty there that needs to be done.}}{Alan Turing}{}

In this thesis, our main objective was to address two complex combinatorial
optimization problems from the Google Hash Code competition using~\acrshort{meta-heuristic}
approaches.~Our approach was centered on developing conceptual models for these
problems and implementing them within a framework we designed. This enabled the
utilization of generic~\acrshort{meta-heuristic} solvers that were also developed to
conform to the framework's specification. Essentially, the framework acted as a
practical tool to bridge the gap between conceptual problem modeling and actual
problem-solving, while offering a platform for creating generic~\acrshort{meta-heuristic}
solvers that can tackle any problem in a black-box fashion.

Our efforts resulted in successful problem modeling and promising outcomes for
both of the problems we attempted to solve. Notably, all the problem models we
devised could be implemented through the framework, highlighting the
adaptability and effectiveness of our approach in tackling intricate problems.
Moreover, our framework allowed the development of all the~\acrshort{meta-heuristic}
strategies discussed in~\Cref{ch:background}, demonstrating that is indeed
possible to this. However, it's important to acknowledge that the framework we
constructed will benefit from further refinement, especially in terms of
accommodating different modeling aspects, such as describing models for other
types of meta-heuristics like evolutionary algorithms, which were not within the
scope of this work.

Overall, the experience of working with this framework has positive, primarily
due to its transparency in addressing essential modeling questions and its clear
separation of solvers and models, making a wide range of utilities readily
available. However, becoming proficient in the framework requires
training and conceptual familiarity, which will only come with practice.

\section{Future Work}

Possible research directions in future work are presented as follows:

\begin{description}
  \item[Problem Models] Expanding the repertoire of problem models, both for
    the existing Google Hash Code problems and for other challenges within the
    competition, holds significant importance. This broader range of models will
    enable a comprehensive assessment of how the framework performs when applied
    to problems with diverse characteristics. The models we've constructed so far
    represent just a small subset, and a more extensive selection will facilitate
    a more in-depth analysis of~\acrshort{meta-heuristic} behavior. This expansion is
    crucial, as having a variety of performant models will support extensive
    testing and refinement of solvers.

  \item[Meta-Heuristic Implementation] Further implementations of other
    meta-heuristic techniques, such as evolutionary algorithms, can be
    explored. The introduction of new meta-heuristics may necessitate updates
    and additional functionalities for the framework to accommodate these approaches effectively.

  \item[Experimental Evaluation of Meta-Heuristics] The results obtained from the
    developed models have not yet delved into the parametrization of meta-heuristic
    methods. This analysis is crucial for optimizing and fine-tuning these
    techniques to achieve the best possible performance in solving complex problems.
    Furthermore, future work should include a thorough examination and comparison
    of different  meta-heuristics.
\end{description}
\chapter{Conclusion}
\label{ch:conclusion}

\chquote{\textquote{We can only see a short distance ahead, but we can see plenty there that needs to be done.}}{Alan Turing}{}

In this thesis, our main objective was to address two complex combinatorial
optimization problems from the Google Hash Code competition using~\acrshort{meta-heuristic}
approaches.~Our approach was centered on developing conceptual models for these
problems and implementing them within a framework we designed. This enabled the
utilization of generic~\acrshort{meta-heuristic} solvers that were also developed to
conform to the framework's specification. Essentially, the framework acted as a
practical tool to bridge the gap between conceptual problem modeling and actual
problem-solving, while offering a platform for creating generic~\acrshort{meta-heuristic}
solvers that can tackle any problem in a black-box fashion.

Our efforts resulted in successful problem modeling and promising outcomes for
both of the problems we attempted to solve. Notably, all the problem models we
devised could be implemented through the framework, highlighting the
adaptability and effectiveness of our approach in tackling intricate problems.
Moreover, our framework allowed the development of all the~\acrshort{meta-heuristic}
strategies discussed in~\Cref{ch:background}, demonstrating that is indeed
possible to this. However, it's important to acknowledge that the framework we
constructed will benefit from further refinement, especially in terms of
accommodating different modeling aspects, such as describing models for other
types of meta-heuristics like evolutionary algorithms, which were not within the
scope of this work.

Overall, the experience of working with this framework has positive, primarily
due to its transparency in addressing essential modeling questions and its clear
separation of solvers and models, making a wide range of utilities readily
available. However, becoming proficient in the framework requires
training and conceptual familiarity, which will only come with practice.

\section{Future Work}

Possible research directions in future work are presented as follows:

\begin{description}
  \item[Problem Models] Expanding the repertoire of problem models, both for
    the existing Google Hash Code problems and for other challenges within the
    competition, holds significant importance. This broader range of models will
    enable a comprehensive assessment of how the framework performs when applied
    to problems with diverse characteristics. The models we've constructed so far
    represent just a small subset, and a more extensive selection will facilitate
    a more in-depth analysis of~\acrshort{meta-heuristic} behavior. This expansion is
    crucial, as having a variety of performant models will support extensive
    testing and refinement of solvers.

  \item[Meta-Heuristic Implementation] Further implementations of other
    meta-heuristic techniques, such as evolutionary algorithms, can be
    explored. The introduction of new meta-heuristics may necessitate updates
    and additional functionalities for the framework to accommodate these approaches effectively.

  \item[Experimental Evaluation of Meta-Heuristics] The results obtained from the
    developed models have not yet delved into the parametrization of meta-heuristic
    methods. This analysis is crucial for optimizing and fine-tuning these
    techniques to achieve the best possible performance in solving complex problems.
    Furthermore, future work should include a thorough examination and comparison
    of different  meta-heuristics.
\end{description}
\chapter{Conclusion}
\label{ch:conclusion}

\chquote{\textquote{We can only see a short distance ahead, but we can see plenty there that needs to be done.}}{Alan Turing}{}

In this thesis, our main objective was to address two complex combinatorial
optimization problems from the Google Hash Code competition using~\acrshort{meta-heuristic}
approaches.~Our approach was centered on developing conceptual models for these
problems and implementing them within a framework we designed. This enabled the
utilization of generic~\acrshort{meta-heuristic} solvers that were also developed to
conform to the framework's specification. Essentially, the framework acted as a
practical tool to bridge the gap between conceptual problem modeling and actual
problem-solving, while offering a platform for creating generic~\acrshort{meta-heuristic}
solvers that can tackle any problem in a black-box fashion.

Our efforts resulted in successful problem modeling and promising outcomes for
both of the problems we attempted to solve. Notably, all the problem models we
devised could be implemented through the framework, highlighting the
adaptability and effectiveness of our approach in tackling intricate problems.
Moreover, our framework allowed the development of all the~\acrshort{meta-heuristic}
strategies discussed in~\Cref{ch:background}, demonstrating that is indeed
possible to this. However, it's important to acknowledge that the framework we
constructed will benefit from further refinement, especially in terms of
accommodating different modeling aspects, such as describing models for other
types of meta-heuristics like evolutionary algorithms, which were not within the
scope of this work.

Overall, the experience of working with this framework has positive, primarily
due to its transparency in addressing essential modeling questions and its clear
separation of solvers and models, making a wide range of utilities readily
available. However, becoming proficient in the framework requires
training and conceptual familiarity, which will only come with practice.

\section{Future Work}

Possible research directions in future work are presented as follows:

\begin{description}
  \item[Problem Models] Expanding the repertoire of problem models, both for
    the existing Google Hash Code problems and for other challenges within the
    competition, holds significant importance. This broader range of models will
    enable a comprehensive assessment of how the framework performs when applied
    to problems with diverse characteristics. The models we've constructed so far
    represent just a small subset, and a more extensive selection will facilitate
    a more in-depth analysis of~\acrshort{meta-heuristic} behavior. This expansion is
    crucial, as having a variety of performant models will support extensive
    testing and refinement of solvers.

  \item[Meta-Heuristic Implementation] Further implementations of other
    meta-heuristic techniques, such as evolutionary algorithms, can be
    explored. The introduction of new meta-heuristics may necessitate updates
    and additional functionalities for the framework to accommodate these approaches effectively.

  \item[Experimental Evaluation of Meta-Heuristics] The results obtained from the
    developed models have not yet delved into the parametrization of meta-heuristic
    methods. This analysis is crucial for optimizing and fine-tuning these
    techniques to achieve the best possible performance in solving complex problems.
    Furthermore, future work should include a thorough examination and comparison
    of different  meta-heuristics.
\end{description}

\backmatter

\printglossary[type=\acronymtype]
\printbibliography[heading=bibintoc]

\end{document}