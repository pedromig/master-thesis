In this chapter, we presented our analysis and described the model for the
Google Hash Code problem entitled~\textquote{\nameref{subsubsec:hashcode-2020-qualification}}
This problem posed significant challenges, both in terms of conceptualization
and implementation using the modeling framework.

From a conceptual perspective, developing the model for this problem involved a
great deal of thought about possible approaches to address the issues arising
from the large instance sizes. Many ideas were tested, and most of them yielded
poor results due to either slowness or worse performance on specific problem
instances with more complex characteristics. Nevertheless, this problem
underscored the importance of creating a highly descriptive model and
demonstrated that a well-crafted model can significantly impact the
effectiveness of problem-solving algorithms.

From an implementation perspective, our experience showed that the framework is
flexible enough to accommodate various problem-solving approaches. It provides a
versatile~\acrshort{api} that gives users the freedom to employ different strategies for
solving problems while adhering to the framework's specifications. This
flexibility allows users to leverage existing solvers and experiment with new
algorithms easily.