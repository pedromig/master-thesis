This chapter, described our model and analysis of the google hash code problem
entitled~\textquote{\nameref{subsubsec:hashcode-2020-qualification}}.
This problem presented instances, each with different challenges and
relatively large sizes.

We began by approaching the problem from a constructive search perspective, and
in doing so, we made several choices related to the enumeration of components
and the calculation of the upper bound. Initially, we attempted to use the~\emph{standard}
enumeration method for components, but it quickly became apparent that this
approach was impractical due to the size of the instances. Consequently, we
switched to the~\emph{sequential} enumeration method, which was better suited
for handling the large instances and ultimately became our default choice.
Concerning the upper bound, the first one presented in the problem model proved
to be too computationally demanding and was not used in practice. However, the
second bound, which further relaxed the constraints of the problem, proved to be
more performant and is the one we utilized in practice.

These choices enabled us to employ a greedy
constructive~\acrshort{meta-heuristic} that selected the solution with the best
upper bound value at each step with a 15 minute timeout, leading to the results
presented in~\Cref{tab:bs-cs-results}.~Other experiments were conducted with
various constructive meta-heuristics we developed in the context of the modeling
framework. However, these approaches proved to be slower in optimizing the
solution within the aforementioned time frame.

\begin{table}[h]
  \centering
  \begin{tabular}{@{\extracolsep{4pt}}lc@{\extracolsep{4pt}}}
    \toprule
    Instance                           & Score         \\ \midrule
    \textquote{Read On}                & \num{5822900} \\
    \textquote{Incunabula}             & \num{185017 } \\
    \textquote{Tough Choices}          & \num{2665}    \\
    \textquote{So many books}          & \num{140189}  \\
    \textquote{Libraries of the world} & \num{275446}  \\
    \bottomrule
  \end{tabular}
  \caption{Book Scanning Simple Constructive Search Results}
  \label{tab:bs-cs-results}
\end{table}

From a local search perspective, we encountered challenges in achieving improved
results using the implemented meta-heuristics. Despite having promising local
moves in theory, none of them led to improvements in the solution obtained after
constructive search. Consequently, we considered an alternative approach that
focused on optimizing the ordering of libraries and then solved the book
assignment problem exactly.

For this purpose, we employed the two-phase approach which yielded the best
results we were able to achieve. Remarkably, the usage of an exact solver for
the~\emph{max-flow min-cost} problem provided in the \emph{Python}
package~\emph{networkx}\footnote{\url{https://networkx.org/}} achieved
surprisingly good performance, taking at most 2 minutes to fully construct a
solution after an ordering for the libraries was established, even in the
largest instance.~Furthermore, the local search following this construction,
using a~\acrshort{hill-climbing} technique, was able to further improve the
scores for some of the instances within the 30-minute time limit we set for this
procedure.

This approach indeed yielded the best results we were able to achieve, which are
presented in detail in~\Cref{tab:bs-results} alongside the best results obtained
by other contestants during the competition.

\begin{table}[h]
  \centering
  \begin{tabular}{@{\extracolsep{4pt}}lcc@{\extracolsep{4pt}}}
    \toprule
    Instance                           & Score         & Best Known Score \\ \midrule
    \textquote{Read On}                & \num{5822900} & \num{5822900}    \\
    \textquote{Incunabula}             & \num{5689822} & \num{5690888}    \\
    \textquote{Tough Choices}          & \num{5028530} & \num{5107113}    \\
    \textquote{So many books}          & \num{5208455} & \num{5237345}    \\
    \textquote{Libraries of the world} & \num{5328034} & \num{5348248}    \\
    \bottomrule
  \end{tabular}
  \caption{Book Scanning Best Results}
  \label{tab:bs-results}
\end{table}

Remarkably, this placed us in~\textbf{33rd} place on the competition leaderboard.