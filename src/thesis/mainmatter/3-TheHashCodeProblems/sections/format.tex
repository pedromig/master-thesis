The Google Hash Code competition, organized by Google from 2014 to 2022,
consisted of two main phases: a qualifying round and a final round. During this
competition, teams of 2-4 skilled individuals were tasked with solving complex
problems that mirrored real-world engineering challenges faced by Google's own
engineers. The primary aim of the competition was to attract talented
individuals to the company. In the qualifying round, participants worldwide
engaged in a 4-hour problem-solving session. Subsequently, around 40-50 select
teams advanced to the final round, which took place at a Google headquarters.
Additionally, participants gathered at designated hubs globally during the
qualifying round, fostering a competitive environment.

\begin{figure}[ht]
  \centering
  \begin{tikzpicture}

  % Data
  \pgfplotstableread{
    Label Qualification Final
    2014 100 54
    2015 230 65
    2016 1050 51
    2017 2815 50
    2018 3012 40
    2019 6672 41
    2020 10725 45
    2021 9003 38
    2022 10177 39
  }\data

  % Legend Style
  \pgfplotsset{
    /pgfplots/area legend/.style={
        /pgfplots/legend image code/.code={
            \fill[##1] (0cm,0.6em) rectangle (2*\pgfplotbarwidth,-0.3em);
          },
      },
  }

  \begin{axis}[
      xlabel = {Year},
      ylabel = {Teams},
      ybar,
      ymin=0,
      opacity=0.8,
      xtick=data,area legend,
      ymode=log,
      width=0.8\textwidth,
      height=\axisdefaultheight,
      legend style={cells={anchor=west}, legend pos=north west, font=\scriptsize},
      x tick label style={font=\scriptsize},
      y tick label style={font=\scriptsize},
      xticklabels from table={\data}{Label},
      xticklabel style={text width=2cm,align=center},
    ]
    % Qualification Round
    \addplot [fill=cb_dark_blue] table [y=Qualification, meta=Label, x expr=\coordindex] {\data};
    \addlegendentry{Qualification Round}

    % Final Round
    \addplot [fill=cb_orange] table [y=Final, meta=Label, x expr=\coordindex] {\data};
    \addlegendentry{Final Round}
  \end{axis}
\end{tikzpicture}
  \caption{Google Hash Code Competition Attendance 2014-2022}
  \label{fig:hashcode-attendance}
\end{figure}

In the early years of the competition, it was only open to teams from France. In
the subsequent three years, it was open to teams from Europe, Africa, and the
Middle East before becoming a worldwide competition. Therefore, it is expected
that there will be an increase in the number of results available in the later
years and more challenging problems due to the increase in competition. Furthermore,
the number of participants kept growing throughout the years which highlights
the importance of this event as illustrated in~\Cref{fig:hashcode-attendance}

Unfortunately, this year Google decided to cease all its coding competitions
including Hash Code. Nevertheless, the competition generated a diverse
collection of attempts at solving the problems, resulting in a valuable wealth
empirical data accessible to the community. As the official coding competition
website is no longer accessible, Google created a repository containing all
problem statements and instances distributed under an open source license
~\cite{llc2023codingcompetitionsarchive}. However, it's worth noting that the
scores achieved by participants are not integrated into this repository.
Instead, they are documented across various blog posts and third-party
websites~\cite{googlea}.

% TODO: e.g Simulated Annealing
It is worth noting, that due to the nature of the problems and format of the
competition, participants frequently made use of heuristic and meta-heuristic
strategies to solve the problems as best as possible in the alloted time.
Moreover, the majority of competition problems are structurally different from
one other, which makes it demanding to write general-purpose heuristic solvers
that can be easily reused. Hence, it's a common practice for competitors to use
solvers that are  easily implementable or readily available online, given that
internet access is permitted.