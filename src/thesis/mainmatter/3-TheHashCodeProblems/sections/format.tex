The Google Hash Code competition, organized by Google from 2014 to 2022,
consisted of two main phases: a qualifying round and a final round. During this
competition, teams of 2-4 skilled individuals were tasked with solving complex
problems that mirrored real-world engineering challenges faced by Google's own
engineers. The primary aim of the competition was to attract talented
individuals to the company. In the qualifying round, participants worldwide
engaged in a 4-hour problem-solving session. Subsequently, around 40-50 select
teams advanced to the final round, which took place at a Google headquarters.
Additionally, participants gathered at designated hubs globally during the
qualifying round, fostering a competitive environment.

\pgfplotstableread{
  Label Qualification Final
  2014 100 54
  2015 230 65
  2016 1050 51
  2017 2815 50
  2018 3012 40
  2019 6672 41
  2020 10725 45
  2021 9003 38
  2022 10177 39
}\testdata

\begin{figure}[h]
  \centering
  \begin{tikzpicture}
    \begin{axis}[
        xlabel = {Year},
        ylabel = {Participants},
        ybar,
        ymin=0,
        opacity=0.8,
        xtick=data,
        ymode=log,
        width=0.8\textwidth,
        height=\axisdefaultheight,
        legend style={cells={anchor=west}, legend pos=north west, font=\footnotesize},
        x tick label style={font=\footnotesize},
        y tick label style={font=\footnotesize},
        reverse legend=false,
        xticklabels from table={\testdata}{Label},
        xticklabel style={text width=2cm,align=center},
      ]
      \addplot [fill=cb_blue] table [y=Qualification, meta=Label, x expr=\coordindex] {\testdata};
      \addlegendentry{Qualification Round}
      \addplot [fill=cb_orange] table [y=Final, meta=Label, x expr=\coordindex] {\testdata};
      \addlegendentry{Final Round}
    \end{axis}
  \end{tikzpicture}
  \caption{Google Hash Code Competition Attendance 2014-2022}
  \label{fig:hashcode-attendance}
\end{figure}


In the early years of the competition, it was only open to teams from Paris. In
the subsequent three years, it was open to teams from Europe, Africa, and the
Middle East before becoming a worldwide competition. Therefore, it is expected
that there will be an increase in the number of results available in the later
years and more challenging problems due to the increase in competition. Futhermore,
the number of participants kept growing throughout the years which highlights
the importance of this event as illustrated in~\Cref{fig:hashcode-attendance}

Unfortunately this year, the Hash Code competition, along with other coding
competitions, ceased due to significant layoffs. Nevertheless, the competition
generated a diverse collection of attempts at solving the problems, resulting in
a valuable wealth empirical data accessible to the community.

Lastly, due to the competition format, it is worth mentioning that competitors
frequently make use of greedy and other heuristic strategies tailored to the
challenge. Moreover, meta-heuristics in this situation are not as popular due to
the time constraints imposed by the competition format. At first glance, the
majority of optimization problems are structurally different from one other,
which makes it demanding to write general-purpose heuristic solvers that can be
easily reused.  On the other hand, it might not be worth the effort spent in the
implementation of such solvers, particularly as the benefit of using a simple
heuristic strategy may outweigh the development and the running cost of
meta-heuristic solvers such as evolutionary algorithms.