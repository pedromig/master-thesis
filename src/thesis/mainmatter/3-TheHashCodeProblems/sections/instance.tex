In the competition context, in combination with the problem statements, test
case instances are furnished to participants with the primary aim of evaluating
the efficacy of their strategies. These instances are carefully generated to
conform to the stipulated limits and constraints inherent to the challenge, as
expounded upon in the problem statement.

The initial instance, commonly denoted as the~\textquote{example}, is routinely
included within the problem statement for contestants' reference. While this
instance is not solved optimally in the statement, its purpose lies in shedding
light on the underlying properties of the problem and furnishing initial
insights that facilitate the formulation of prospective solutions. Notably, this
particular instance is intentionally designed with small dimensions, rendering
approachable via brute force methodologies that are quickly able to solve it
optimally.

Following our observations, the subsequent instances are purposefully designed
to push the boundaries of the problem's scope. These instances are intentionally
larger and structured to discourage brute force methods and general heuristics,
aiming to thoroughly examine various aspects of the problem. The ruggedness of
their objective space introduces challenges for search strategies, potentially
rendering some of them ineffective or even unusable within the available time
budget.

Nonetheless, participants have the flexibility to develop focused strategies,
employ various solvers, or utilize custom (heuristic) approaches for individual
instances, despite their intentional design. In fact, in the competition
context, teams are allowed to provide unique solutions for each instance, thus
becoming a common practice among participants to conduct thorough cross-instance
analysis. This practice proves valuable in revealing patterns that can
offer insights into tackling the challenge with greater efficiency.

Finally, given the clearly articulated problem statements and the transparent
instance generation process, participants have the ability to create customized
test instances. This capability proves valuable in practical implementations and
further solidifies these problems as effective benchmarking tools in the field
of optimization.