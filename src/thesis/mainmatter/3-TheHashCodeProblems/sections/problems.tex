\subsection{Hash Code 2014}
\label{subsec:hashcode-2014}

\subsubsection*{Street View Routing}
\label{subsubsec:hashcode-2014-final}

In the context of constructing street view maps there is a need to collect
imagery that is taken by specialized vehicles equipped for that purpose. This
constitutes a challenging problem since given a fleet of cars which may only be
available for a limited amount of time a route for each must be defined as to
maximize the number of streets photographed. City streets are modelled as a
graph where nodes are junctions and the edges are streets connecting said
junctions. Moreover, streets are defined by three distinct properties:
direction, length and cost that will take for the car to traverse the street.

The challenge consists of scheduling the routes for street view cars in the
city, adhering to a pre-determined time budget. The goal is to optimize the
solution by maximizing the sum of the lengths of the traversed streets, while
minimizing the overall time expended in the process. The quality of the solution
for this problem is evaluated by using the sum of the lengths of the streets as
the primary criterion and the time spent as a tie-breaker.

The problem at hand bears a strong resemblance to a combination of the Vehicle
Routing Problem and the Maximum Covering Problem. This is because the scheduling
of routes for the fleet of cars must be done in a way that ensures that the
combination of all sets of streets visited by each car encompasses the entire
city, in the most time-efficient manner possible.

\subsection{Hash Code 2015}
\label{subsec:hashcode-2015}

\subsubsection*{Optimize a Data Center}
\label{subsubsec:hashcode-2015-qualification}

The optimization of server placement problem is a concern that pertains to the
design of data centers, as various factors must be taken into account to ensure
optimal efficiency. In this context, the `optimizing servers'' problem portrays
a scenario in which contestants are in the position of designing a data center
and seeking to determine the optimal distribution of servers. The data center
is physically organized in rows of slots where servers can be placed. Hence,
the challenge is to efficiently fill the available slots in a Google data center
with servers of varying sizes and computing capacities, while also ensuring that
each server is assigned to a specific resource pool.

Objectively, the goal is to assign multiple servers to available slots and
resource pools in such a way as to maximize the guaranteed capacity for all
resource pools. This metric serves as the criterion for evaluating solutions to
this problem. The guaranteed capacity, in this context, refers to the lowest
amount of computing power that will remain for a specific resource pool in the
event of a failure of an arbitrary row of the data center. It is important to
note that this objective function is considered a bottleneck, as small changes
in a solution may not result in significant changes in the score, making the
optimization process more difficult.

Notably, the problem of optimizing the placement of servers in a data center can
be thought of as a combination of a Multiple-Knapsack Problem and an assignment
problem. This is because the servers must be placed within the constraints of
the available space in the data center rows, and subsequently, they must be
assigned to resource pools.

\subsubsection*{Loon}
\label{subsubsec:hashcode-2015-final}

Project \textit{Loon}, which was a research endeavor undertaken by Google, aimed at
expanding internet coverage globally by utilizing high altitude balloons. The
problem presented in this competition drew inspiration from this concept,
requiring contestants to devise plans for position adjustments for a set of
balloons, taking into consideration various environmental factors, particularly
wind patterns, with the objective of ensuring optimal internet coverage in a
designated region over a specific time frame.

The objective of this problem was to develop a sequence of actions, including
ascent, descent, and maintaining altitude, for a set of balloons with the goal
of maximizing a score. In this case, the score is calculated based on the
aggregate coverage time of each location, represented as cells on a map of
specified dimensions, at the conclusion of the available time budget.

In summary, this problem can be classified as both a simulation and a coverage
and routing problem, based on the properties previously described. It is
important to note that the simulation aspect of this problem has a direct impact
on the calculation of the score, and is not solely limited to constraints on the
available time budget for operations. Furthermore, this problem can be
represented in a forest, where the vertices represent spatiotemporal coordinates
$(x, y, z, t)$, and the edges symbolize changes in altitude and lateral movement
(wind) for a given balloon.

\subsection{Hash Code 2016}
\label{subsec:hashcode-2016}

\subsubsection*{Delivery}
\label{subsubsec:hashcode-2016-qualification}

In today's world, with the widespread availability of internet, online shopping
has become a prevalent activity. As a consequence, there is an ever-growing need
for efficient delivery systems. This competition challenges participants to
manage a fleet of drones, which are to be used as vehicles for the distribution
of purchased goods. Given a map with delivery locations, a set of drones, each
with a set of operations that can be performed (load, deliver, unload, wait), a
number of warehouses, and a number of orders, the objective is to satisfy the
orders in the shortest possible time, taking into consideration that the
products to be delivered in an order may have product items stored in multiple
different warehouses and therefore require separate pickups by drones.

In this problem, the simulation time $\mathcal{T}$ is given and the goal is to
complete each order within that time frame. The score for each order is
calculated as $\frac{(\mathcal{T} - t)}{\mathcal{T}} \times 100$, where $t$ is
the time at which the order is completed. The score ranges from 1 to 100, with
higher scores indicating that the order was completed sooner. The overall score
for the problem is the sum of the individual scores for all orders, and it is to
be maximized.

In summary, this problem can be classified as a variant of the Vehicle Routing
Problem, specifically as a Capacitated, Pickup and Delivery Time Windowed
Multi-Depot Vehicle Routing Problem. This classification takes into account the
pickup and delivery of items, the time window for delivery, the multiple routes
and warehouses that each vehicle may need to visit in order to fulfill the
orders.

\subsubsection*{Satellites}
\label{subsubsec:hashcode-2016-final}

\textit{Terra Bella} was a Google division responsible for managing and
operating a constellation of satellites that collected and processed imagery for
commercial purposes. Specifically, these satellites were tasked with capturing
images in response to client requests.

The challenge presented to participants involves crafting schedules for
individual satellites within the fleet. The goal is to secure image collections
that match customer preferences. These collections are characterized by
geographical coordinates on Earth and specific time windows for image capture.
Each satellite, originating from unique latitude and longitude coordinates and
possessing a certain velocity, possesses the ability to make minor positional
adjustments along both axes to access potential photography sites. The problem's
score is determined by aggregating the points earned through the successful
completion of customer collections. In this context, completion signifies
capturing all images for a given collection within the designated time frame.

In essence, this problem falls into the categories of both an assignment and a
maximum covering problem. It involves not only covering the maximum number of
images with the available satellites to complete collections, but also making
decisions about which satellites will capture each photo. Additionally, the
simulation aspect is crucial as it directly affects scoring; images not taken
within the specified time frame won't contribute to the collection, potentially
influencing its completion and the overall score.

\subsection{Hash Code 2017}
\label{subsec:hashcode-2017}

\subsubsection*{Streaming Videos}
\label{subsubsec:hashcode-2017-qualification}

In the era of online streaming services like YouTube, effectively distributing
content to users is crucial. This challenge focuses on optimizing video
distribution across cache servers to minimize transmission delays and waiting
times for users. Contestants must strategise video placement within servers
while considering space limitations.

With a roster of videos, each assigned a specific size, an collection of cache
servers with designated space, and an index of endpoints initiating multiple
requests for various videos, this challenge entails determining an optimal
video assignment within servers. The time saved for each request is measured as the
difference between data center streaming time and cache server streaming time
with minimal latency. The overall score is computed by summing the time saved
for each request, multiplied by 1000, and then divided by the total request
count. It's important to note that the problem description offers transmission
latencies between different nodes.

In general, we categorize this problem as a combination of assignment and
knapsack problems. Contestants are tasked not only with determining the
allocation of videos to servers but also with accounting for capacity
limitations on the number of videos per server. It's worth noting that the
calculation of time saved for each request may encounter a bottleneck effect,
which can pose challenges when optimizing the overall score.

\subsubsection*{Router Placement}
\label{subsubsec:hashcode-2017-final}

Strategically optimizing the placement of Wi-Fi routers to achieve optimal
signal coverage is a challenge encountered by many institutions and individual
users. This issue becomes particularly prominent in larger and complex
buildings. Furthermore, in such scenarios, the task may involve setting up a wired
connection to establish internet connectivity from the source point,
facilitating the strategic positioning of routers for maximum coverage.

The challenge tasked participants with optimizing the arrangement of routers and
fiber wiring within a building's cell-based layout, along with a designated
backbone connection point. The aim was to strategically position routers and
devise an effective wiring configuration. The primary goal encompassed achieving
optimal coverage while adhering to a predefined budget. The problem's score
comprised two components: the count of cells covered by routers, multiplied by
1000, and the remaining budget. Notably, the scoring approach emphasized both
extensive coverage and economical budget allocation.

In essence, this problem falls under the category of a maximum covering problem,
as the central aim is to ensure the coverage of as many cells as possible.
Furthermore, considering the budget limitations and wiring arrangement, we
observe that this challenge shares similarities with the Steiner Tree Problem.
This likeness arises from the possibility of determining the optimal cost of
wiring placement based on the router locations, which may hold significance for the
problem's resolution.

\subsection{Hash Code 2018}
\label{subsec:hashcode-2018}

\subsubsection*{Self-Driving Rides}
\label{subsubsec:hashcode-2018-qualification}

Daily car commuting is a ubiquitous practice globally, involving trips to homes,
schools, workplaces, and more. As a means of travel, cars remain a common
choice, with ongoing efforts to enhance safety through the advancement of
self-driving technology. In this challenge, contestants assume the role of
managing a fleet of self-driving cars within a simulated setting. The goal is to
ensure commuters reach their destinations securely and punctually.

With a fleet of cars at disposal and a roster of rides defined by their starting
and ending intersections on a square grid representing the city, along with the
earliest start time and the latest end time to ensure punctuality, the task is
to allocate rides to vehicles. The aim is to maximize the number of completed
rides before a predefined simulation time limit is reached. The scoring is
determined by the summation of the individual ride scores. A ride's score is
computed as the sum of a value proportional to the distance covered during the
ride, augmented by a bonus if the ride commences precisely at its earliest
allowed start time.

Generally, this problem can be categorized as an assignment and vehicle routing
problem with time windows. This classification arises from the necessity to
assign rides to cars within specific time constraints. Notably, the car's route
is determined by the sequence of rides assigned to it. Moreover, this challenge
falls under the simulation category, as it directly impacts the scoring
mechanism and cannot be simplified or abstracted.

\subsubsection*{City Plan}
\label{subsubsec:hashcode-2018-final}

With the world's population increasingly concentrating in urban areas, the
demand for expanded city infrastructure is on the rise. This entails not only
residential buildings but also the incorporation of essential public facilities
and services to cater to the growing populace. This challenge mirrors a scenario
where participants are tasked with planning a city's building layout,
involving both the selection of building types and their strategic placement.

For this challenge, participants receive building projects with specific width and
height dimensions, covering both residential and utility structures. The city is
a square grid of cells. Overall, the goal is to create buildings from these
plans, arranging them within the city to optimize space and create a balanced
mix of structures. This minimizes residents' walking distance to reach essential
services. The overall score is the sum individual residential building scores,
calculated by multiplying the number of residents and the number of utility
building types within walking distance of that building. Notably, the walking
distance parameter is specific to each problem instance.

Essentially, this problem belongs to the category of packing problems. The core
objective revolves around determining how to fit buildings within the city
layout. Importantly, there is no predetermined limit on the number of buildings
that can be constructed for each plan, granting contestants the flexibility to
make choices accordingly.

\subsection{Hash Code 2019}
\label{subsec:hashcode-2019}

\subsubsection*{Photo Slideshow}
\label{subsubsec:hashcode-2019-qualification}

Given the surge in digital photography and the vast number of images traversing
the internet daily, this challenge delves into the interesting concept of crafting
picture slideshows using the available photo pool.

In this scenario, participants were tasked with creating a slideshow composed of
pictures, which could be oriented either vertically or horizontally in the
slides. Notably, a slide could contain two photos if they were arranged
vertically. Additionally, these photos could be tagged with multiple descriptors
corresponding to their subjects. The scoring of this problem revolves around the
slideshow's appeal, determined by a calculated value that depends on consecutive
slide pairs. This value is computed as the minimum between the tags count of the
first picture, the second picture in the sequence and the count of the common
tags shared between the two images.

Overall, this challenge can be categorized as a scheduling problem, to be
precise, a single-machine job scheduling problem. If we liken the jobs to
photos, the goal is to sequence them to optimize a specific objective function
in this context, the~\textquote{appeal} factor. Additionally, the interactions
between slides introduce elements resembling a grouping problem.

\subsubsection*{Compiling Google}
\label{subsubsec:hashcode-2019-final}

Given Google's extensive codebase spanning billions of lines of code across
numerous source code files, compiling these files on a single machine would be
time-consuming. To address this, Google distributes the compilation process
across multiple servers.

This challenge tasks participants with optimizing compilation time by
strategically distributing source code files across available servers. Notably,
the compilation of a single code file can depend on other files being compiled
prior to it, involving dependencies. Given a certain number of available servers
and specific deadlines for compilation targets, the problem's score is
calculated by summing the scores for the completion of each compilation target.
These scores are determined by a fixed value for meeting the deadline, with an
additional bonus if the compilation is completed ahead of the expected time.

This problem can be categorized as a scheduling problem, as the primary
objective involves distributing compilation tasks (jobs) among different
machines while adhering to dependencies between files. In essence, this problem
resembles a variation of the classical job-shop scheduling problem.

\subsection{Hash Code 2020}
\label{subsec:hashcode-2020}

\subsubsection*{Book Scanning}
\label{subsubsec:hashcode-2020-qualification}

\textit{Google Books} is project that aims to create a digital collection of many books
by scanning them from libraries and publishers around the world. In this
challenge, contestants are put in the position of managing the operation of
setting up a scanning pipeline for millions of books.

Given a dataset describing libraries and available books, the objective of this
challenge is to select books for scanning from each library within a specified
global deadline. Each library has a distinct sign-up process duration before it can
commence scanning, and only one library can be signed up at a time. Moreover,
each library has a fixed scanning rate for books per day, and each scanned book
contributes to the final score. The problem's goal is to maximize the overall
score, which is calculated as the sum of the scores for unique books scanned
within the given deadline.

This problem exhibits a combination of characteristics from classical
scheduling, assignment, covering, and knapsack problems. It resembles a
scheduling problem as the order in which libraries are signed up needs to be
determined. It involves assignment, since libraries can share books,
necessitating a decision on which libraries will scan each book. The covering
aspect is apparent in the scoring mechanism, where the aim is to maximize the
number of unique books scanned.  Lastly, the problem also incorporates a
knapsack-like element. While the time-related simulation factor exists, it can
be abstracted into a knapsack scenario where the goal is to optimize the overall
score by considering the number of books a library can scan until the deadline
as its capacity.

\subsubsection*{Assembling Smartphones}
\label{subsubsec:hashcode-2020-final}

Constructing smartphones is a intricate process that entails assembling a
multitude of hardware components. This challenge delves into the concept of
creating an automated assembly line for smartphones, employing robotic arms to
streamline the manufacturing process.

Contestants are tasked with placing robotic arms within a workspace depicted as
a rectangular cell grid. The objective is to optimize the arrangement of these
arms to allow the execution of assigned tasks. Each task involves specific
movements that a robotic arm must perform, essentially traversing a designated
number of cells to accomplish the task. Notably, robotic arms cannot cross each
other, necessitating precise task assignment and arm positioning to ensure
unobstructed task execution for all arms. The problem's score is derived from
the summation of scores obtained by successfully completing tasks within the
constraints.

In summary, this challenge falls under the category of both assignment and
scheduling problems since it encompasses the assignment of robotic arms to
suitable positions and the scheduling of tasks across these arms to optimize the
completion of tasks.

\subsection{Hash Code 2021}
\label{subsec:hashcode-2021}

\subsubsection*{Traffic Signaling}
\label{subsubsec:hashcode-2021-qualification}

This challenge delves into the optimization of traffic light timers to enhance
the travel experience in a city. While traffic lights inherently contribute to
road safety, their built-in timers are important in regulating traffic flow. The
focus here is to fine-tune these timers with the aim of optimizing overall
travel time for all commuters within the city.

Contestants are presented with a city layout, complete with intersections
housing traffic lights. The task is to strategically allocate time intervals to
these traffic light timers, optimizing traffic flow to ensure the maximum number
of car trips are successfully completed within a predefined simulation time
limit. The problem's score is the cumulative sum of scores assigned to each
completed trip. These scores comprise a fixed value for trip completion and a
bonus proportional to how early the trip concludes relative to the simulation's
time limit. While the challenge may seem complex due to its detailed rules and
operational aspects, its core objective revolves around this fundamental
optimization process.

In summary, this challenge can be categorized as a simulation problem. It's
worth highlighting that this problem aligns closely with the Signal Timing
problem in the literature of Control Optimization.

\subsubsection*{Software Engineering at Scale}
\label{subsubsec:hashcode-2021-final}

This challenge addresses the complexity of managing Google's vast monolithic
codebase, which has grown significantly alongside the expanding number of
engineers. To overcome the hurdles of effective feature deployment, participants
are tasked with creating a solution that optimally schedules feature
implementation work among engineers.

In this challenge, there are three primary components to be considered: features,
services, and binaries. Each feature may require certain services, which can be
present in specific binaries. The main objective is to efficiently assign
features to engineers, considering that their implementation might entail
additional tasks such as service implementation, binary relocation, new binary
creation, or waiting for a designated time. The challenge revolves around
optimizing this workflow to minimize delays caused by multiple engineers working
in the same service. The scoring is based on the sum of scores awarded for
feature completion. Each completed feature's score is determined by the product
of the number of users benefiting from it, as specified in the problem
statement, and the number of days between the maximum day (also defined) and the
day the feature was launched.

In essence, this challenge falls within the realm of classic scheduling
problems. It involves assigning tasks (jobs) to engineers with the aim of
optimizing a quantity influenced by the order in which each engineer performs
their tasks and the interactions of tasks among multiple engineers.

\subsection{Hash Code 2022}
\label{subsec:hashcode-2022}

\subsubsection*{Mentorship and Teamwork}
\label{subsubsec:hashcode-2022-qualification}

\subsubsection*{Santa Tracker}
\label{subsubsec:hashcode-2022-final}

\subsection{Outline}
\label{subsec:hashcode-outline}

In summary, this section provided an overview and description of the key aspects
of the Hash Code problems. Furthermore, a categorization that links these
problems to topics commonly found in combinatorial optimization literature was
presented. The~\Cref{tab:hashcode-summary} shows a summary of the
analysis conducted.

\begin{table}[ht]
  \centering
  \resizebox{\textwidth}{!}{% 
    \begin{tabular}{@{\extracolsep{4pt}}cccccccc}
      \toprule
      \multirow{2}{*}{\textbf{Problem}} & \multicolumn{7}{c}{\textbf{Categories}}                                                                                    \\
      \cmidrule{2-8}
      {}                                & Assignment                              & Knapsack   & Coverage   & Vehicle Routing & Simulation & Scheduling & Packing    \\ \midrule
      Street View Routing               &                                         &            & \checkmark & \checkmark      &            &            &            \\
      Optimize a Data Center            & \checkmark                              & \checkmark &            &                 &            &            &            \\
      Loon                              &                                         &            & \checkmark & \checkmark      & \checkmark &            &            \\
      Delivery                          &                                         &            &            & \checkmark      &            &            &            \\
      Satellites                        & \checkmark                              &            & \checkmark &                 & \checkmark &            &            \\
      Streaming Videos                  & \checkmark                              & \checkmark &            &                 &            &            &            \\
      Router Placement                  &                                         &            & \checkmark &                 &            &            &            \\
      Self-Driving Rides                & \checkmark                              &            &            & \checkmark      & \checkmark &            &            \\
      City Plan                         &                                         &            &            &                 &            &            & \checkmark \\
      Photo Slideshow                   &                                         &            &            &                 &            & \checkmark &            \\
      Compiling Google                  &                                         &            &            &                 &            & \checkmark &            \\
      Book Scanning                     & \checkmark                              & \checkmark & \checkmark &                 &            & \checkmark &            \\
      Assembling Smartphones            & \checkmark                              &            &            &                 &            & \checkmark &            \\
      Traffic Signaling                 &                                         &            &            &                 & \checkmark &            &            \\
      Software Engineering at Scale     &                                         &            &            &                 &            & \checkmark &            \\
      Mentorship and Teamwork           &                                         &            &            &                 &            & \checkmark &            \\
      Santa Tracker                     &                                         &            &            & \checkmark      &            &            &            \\ \bottomrule
    \end{tabular}
  }
  \caption{Categorization of Google Hash Code Problems}
  \label{tab:hashcode-summary}
\end{table}