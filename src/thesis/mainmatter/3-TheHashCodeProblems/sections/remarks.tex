In this chapter, we conducted a comprehensive exploration of the Google Hash
Code competition, delving into its structure, problem descriptions, and
instances. In particular, we established links between the challenges presented
and well-known~\acrshort{combinatorial-optimization} problems. Furthermore, we
highlighted common techniques employed by participants, drawing from our own
engagement over several years.

We consider this analysis to be an important step that not only facilitated a
deeper comprehension of the challenges, but also guided our choice of two
specific problems (\nameref{subsubsec:hashcode-2015-qualification}
and~\nameref{subsubsec:hashcode-2020-qualification}) for detailed exploration in
this study. The particular choice of these problems is mainly motivated by the
range of combinatorial optimization topics covered, leaving only vehicle routing
and simulation as subjects to address in future work.

Moreover, based on the conducted analysis, we once again emphasize the
importance of these problems as promising candidates for black-box
optimization~\cite{bartz-beielstein2020benchmarking}.  However, we believe that
for this potential to be realized, it is essential to establish a repository
containing the scores achieved across various problems and instances. Ideally,
this repository should be accompanied by the corresponding source code for
reproducibility purposes. From the standpoint of experimentally evaluating
meta-heuristics for these problems, we consider it vital to generate a diverse
set of instances, eventually generated through
different methods.

Having understood the Google Hash Code problems, in the ensuing chapters, we
will discuss our modelling approach to solve them, analyze the chosen problems,
and conclude with a reflection on the work carried out.