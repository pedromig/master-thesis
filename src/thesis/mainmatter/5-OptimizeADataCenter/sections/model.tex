In the following, we delve into the modeling developed for this problem, and
highlight important aspects related to its implementation within the framework
proposed in~\Cref{ch:principled-modelling-framework}.

\subsubsection*{Problem}

A problem instance is uniquely defined by the set of available servers eligible
for allocation, along with their respective sizes and capacities, the number of
resource pools to be created, and the set of segments, each of which is
characterized by its size.

This corresponds to the information stored within the~\texttt{Problem} class.
From an implementation standpoint, certain pre-processing operations were necessary to
obtain this information. In particular, the construction of set of available
segments using information related to available rows and the list of unavailable
slots.~Furthermore, given that this class serves as a repository for immutable
problem-related information and is instantiated only once, a server ordering
relevant for the construction rules was also included.~This enabled us to
iterate through and access the server order without the need to recompute it.

\subsubsection*{Solution}

A solution is characterized by a collection of server assignments to pools and
segments.~Notably, in the context of this problem, any partial solution is
feasible.

This data is contained within the~\texttt{Solution} class, along with other
pertinent information that is helpful in describing the state of the
solution.~This includes details such as the set of unassigned servers, the
guaranteed capacities for each of the pools, the total capacities for each pool,
and the same capacities data broken down by row. Furthermore, it also
encompasses the details of evaluation metrics like the current objective
function and upper bound values, in addition to other relevant data that
facilitates their incremental calculations.

\subsubsection*{Component}

A component can be represented as a tuple that contains a server, a pool, a
segment, effectively representing an assignment.~However, it is important to
note that a valid decision is to refrain from placing a server in any of the
available segments.~As such, another valid component can be characterized solely
by the server, indicating the action of~\emph{forbidding} it from being placed.
Notably, this information is encoded in the~\texttt{Component} class.

\begin{equation}
  \label{eq:odc-components-1}
  \mathcal{C} = \{(m, p, i) \mid \forall m = 1, \ldots, \mathcal{M},\ \forall p = 1, \ldots, \mathcal{P},\ \forall i = 1, \ldots, \mathcal{I}\}
\end{equation}
\begin{equation}
  \label{eq:odc-components-2}
  \mathcal{F} = \{m \mid \forall m = 1, \ldots, \mathcal{M}\}
\end{equation}

Given the component definitions
in~\Cref{eq:odc-components-1,eq:odc-components-2}, the ground set can be defined
as $\mathcal{G} = \mathcal{C} \cup \mathcal{F}$.

\subsubsection*{Construction Rules}

Since not all subsets of components represent feasible solutions, the
construction rules specify which components can be enumerated and the order by
which the enumeration is done. As such, the following sections describe the
approaches that were considered.

\paragraph*{Standard}

This approach involves enumerating all possible combinations of segments and
pools where any of the unassigned servers can be assigned. It also considers the
option of~\emph{forbidding} a server from being
used.~\Cref{algorithm:odc-enum-standard} illustrates the pseudocode for this
enumeration.~Note that, the function~\texttt{IsUsed} denotes whether a given
server is available for assignment. Moreover, the function~\texttt{Fit}
indicates if a server $m$ can be assigned to segment $i$.

\begin{algorithm}[h]
  % Standard Enumeration (algorithm2e pseudocode)

\KwIn{Number of Servers ($\mathcal{M}$), Number of Pools ($\mathcal{P}$), Number of Segments ($\mathcal{I}$)}
\KwOut{Enumerated Components}

\SetKw{Continue}{continue}
\SetKw{Yield}{yield}

\SetKwFunction{Fit}{\texttt{Fit}}
\SetKwFunction{IsAssigned}{\texttt{IsAssigned}}

\For{$m = 1$ \texttt{to} $\mathcal{M}$}{
  \If{\IsAssigned{$m$}}{
    \Continue
  }
  \For{$p = 1$ \texttt{to} $\mathcal{P}$}{
    \For{$i = 1$ \texttt{to} $\mathcal{I}$}{
      \If{\ \Fit{$m, i$}}{
        \Yield $(m, p, i)$ \Comment*[f]{Assign server $m$ to pool $p$ and segment $i$}\;
      }
    }
  }
  \Yield $m$ \Comment*[f]{Forbid server $m$}\;
}
  \caption{Standard Component Enumeration}
  \label{algorithm:odc-enum-standard}
\end{algorithm}


\paragraph*{Sequential}

In this approach, a predetermined order for the servers is followed. The process
involves selecting the next server in a sequence and, similar to the standard
approach, enumerate all possible combinations of segments and pools where a
given server can be assigned. Again, the option for forbidding a server from
being used is considered. The pseudocode for this enumeration is demonstrated in
\Cref{algorithm:odc-enum-sequential}.

\begin{algorithm}[h]
  % Sequential Enumeration (algorithm2e pseudocode)

\KwIn{Server Index ($m$), Server Order ($\phi^\mathcal{M}$), Number of Pools ($\mathcal{P}$), Number of Segments ($\mathcal{I}$)}
\KwOut{Enumerated Components}

\SetKw{Continue}{continue}
\SetKw{Yield}{yield}

\SetKwFunction{Fit}{\texttt{Fit}}
\SetKwFunction{IsAssigned}{\texttt{IsAssigned}}

\If{$\lnot$ \IsAssigned{$\phi_{m}^\mathcal{M}$}}{
  \For{$p = 1$ \texttt{to} $\mathcal{P}$}{
    \For{$i = 1$ \texttt{to} $\mathcal{I}$}{
      \If{\ \Fit{$\phi_{m}^\mathcal{M}, i$}}{
        \Yield $(\phi_{m}^\mathcal{M}, p, i)$\;
      }
    }
  }
  \Yield $\phi_{m}^\mathcal{M}$\;
}
  \caption{Sequential Component Enumeration}
  \label{algorithm:odc-enum-sequential}
\end{algorithm}

Notably, the order considered for server enumeration $\phi^\mathcal{M}$, where
$\phi_m^\mathcal{M}$ denotes the $m$-th server within this order, is inspired by
heuristics for the~\acrshort{knapsack-problem}, and prioritizes servers based on
a non-decreasing order of the ratio between size and capacity, $\ell_{m}/c_{m}$.

\paragraph*{Heuristic}

This approach entails establishing a specific order for servers, pools, and
rows, and then enumerating the potential combinations following that predefined
order.~In terms of pools, the order prioritizes pools with the lowest capacity.
Concerning rows, the order gives preference to rows that have the least capacity
with respect to the pool undergoing enumeration.~Finally, segments are
enumerated based on whether the chosen server can fit into the available space.
It is worth noting that the option of forbidding server placement is also taken
into account in this approach.


In~\Cref{algorithm:odc-enum-heuristic}, the symbol $\phi^{\mathcal{M}}$
represents the previously mentioned order for server enumeration. The symbol
$\phi^\mathcal{P}$ signifies the order for pool enumeration, with
$\phi_{p}^\mathcal{P}$ indicating the $p$-th pool in this order. Finally, the
symbol $\phi^\mathcal{R}$ signifies the order for row enumeration, where
$\phi_{p, r}^\mathcal{R}$ specifies the $r$-th row affected by the choice of the
$p$-th pool in this order.


\begin{algorithm}[h]
  \KwIn{Number of Servers ($\mathcal{M}$), Number of Pools ($\mathcal{P}$), Number of Segments ($\mathcal{I}$),
  Server Order ($\phi^\mathcal{M}$), Pool Order ($\phi^\mathcal{P}$), Row Order ($\phi^\mathcal{R}$)}
\KwOut{Enumerated Components}

\SetKw{Continue}{continue}
\SetKw{Yield}{yield}

\SetKwFunction{Fit}{\texttt{Fit}}
\SetKwFunction{IsAssigned}{\texttt{IsAssigned}}

\For{$m = 1$ \texttt{to} $\mathcal{M}$}{
  \If{\IsAssigned{$\phi_{m}^\mathcal{M}$}}{
    \Continue
  }
  \For{$p = 1$ \texttt{to} $\mathcal{P}$}{
    $z \gets \phi_{p}^\mathcal{P}$\;
    \For{$r = 1$ \texttt{to} $\mathcal{R}$}{
      $k \gets \phi_{z, r}^\mathcal{R}$\;
      \For{$i = 1$ \texttt{to} $\mathcal{I}^{k}$}{
        \If{\ \Fit{$\phi_{m}^\mathcal{M}, i$}}{
          \Yield $(\phi_{m}^\mathcal{M}, \phi_{p}^\mathcal{P}, i)$\;
        }
      }
    }
  }
  \Yield $\phi_{m}^\mathcal{M}$\;
}
  \caption{Heuristic Component Enumeration}
  \label{algorithm:odc-enum-heuristic}
\end{algorithm}

\subsubsection*{Objective Function}

The objective function for this problem remains the same as defined in the
problem description, which involves maximizing the minimum guaranteed capacity
across all pools. However, this function is a bottleneck. Therefore, for the
purpose of optimization, we used an alternative objective function, $g(x)$,
that, for a given assignment, $x$, uses as the objective value the tuple
containing all guaranteed capacities for all the pools sorted in non-decreasing
order, as illustrated in~\Cref{eq:objective-2}.

\begin{equation}
  \label{eq:objective-2}
  \begin{aligned}
    g(x)         =\  & (gc_{\pi_1},\ gc_{\pi_2},\ \ldots,\ gc_{\pi_\mathcal{P}})     \\
    \text{s.t. }     & gc_{\pi_1} \le gc_{\pi_2} \le \ldots \le gc_{\pi_\mathcal{P}}
  \end{aligned}
\end{equation}

In this equation, $\pi$ is a permutation of the set $\{1, 2, \ldots, \mathcal{P}\}$.

\subsubsection*{Upper Bound}

The upper bound developed for this problem is grounded in a relaxation of the
second constraint presented in~\Cref{eq:objective}. Rather than directly
assigning servers to individual segments, we opt to consider the number of slots
available for server placement in each row. This number is determined by summing
the count of available slots across all segments within that row.  Following
this relaxation and for the purpose of computing the upper bound, we refer to
the set containing the number of available slots for placement in all rows
except row $r$ as $\mathcal{W}_{r}$.~The calculation of the initial values for
all elements of this set is displayed in~\Cref{eq:odc-relax}:

\begin{equation}
  \label{eq:odc-relax}
  \mathcal{W}_{r} = \sum_{i \in \mathcal{I}^r} \mathcal{L}_{i} \quad r \in \{1, \ldots, \mathcal{R}\} \setminus \{ r \}
\end{equation}

The upper bound calculation process is then divided into two steps, with the
first step involving the estimation of an optimistic guaranteed capacity,
followed by a second step for correction to ensure a tight upper bound.

In the first step, we consider the total size resulting from every combination
of rows, excluding one row denoted as $\mathcal{W}_r$. We systematically assign
available servers to these rows until the cumulative size is fully utilized. The
sum of server capacities for each row, except one, is then divided by the total
number of pools to derive a quantity referred to as the~\textquote{row-wise upper bound}.

Additionally, the assignment of servers adheres to the previously mentioned
heuristic order ($\phi^\mathcal{M}$), inspired by the knapsack problem. However,
in the context of the upper bound calculation, this order prioritizes servers
that are already assigned, followed by the unassigned servers, and excludes
those marked as forbidden.

Formally, the row-wise upper bound calculation process is outlined
in~\Cref{eq:odc-rowise-bound}:

\begin{equation}
  \label{eq:odc-rowise-bound}
  \begin{aligned}
    \Phi_{ub}^{r} =\  & \frac{\sum_{m = 1}^{\mathcal{M}} c_{\phi_{m}^\mathcal{M}}}{\mathcal{P}}     \\
    \text{s.t. }      & \sum_{m = 1}^{\mathcal{M}} \ell_{\phi_{m}^\mathcal{M}} \leq \mathcal{W}_{r} \\
  \end{aligned}
\end{equation}

In the second step of this calculation, we can further refine each row-wise
upper bound by removing pools and servers that already possess a guaranteed
capacity exceeding the value of the row-wise upper bound when dropping that row.
After removing these elements, we recalculate the bound for the reduced server
set and the number of remaining pools. This process is repeated until no further
corrections can be applied, meaning that no considered pool has a guaranteed
capacity greater than the bound.

The upper bound is determined by selecting the minimum value among all the
row-wise upper bounds after applying the correction, which is calculated, as
illustrated in~\Cref{eq:odc-upper-bound}:

\begin{equation}
  \label{eq:odc-upper-bound}
  \Phi_{ub} = \min_{r = 1}^{\mathcal{R}} \Phi_{ub}^{r}
\end{equation}

While this bound is valuable, it can be enhanced for greater informativeness.
During the computation of each row-wise upper bound, we maintain the smallest
value for each pool in a vector denoted as $\Lambda$. It is important to note
that at the end of this process, the values within $\Lambda$ correspond to an
upper bound on the guaranteed capacity of each pool.~Subsequently, this vector
is utilized to define an upper bound for the problem, as depicted
in~\Cref{eq:odc-upper-bound-2}, where $\pi$ represents a permutation of the set
${1, 2, \ldots, \mathcal{P}}$. Notably, this upper bound aligns with the choice
for the alternative objective function, $g(x)$, mentioned earlier.

\begin{equation}
  \label{eq:odc-upper-bound-2}
  \begin{aligned}
    \Phi_{ub}    =\  & (\Lambda^{\pi_1},\ \Lambda^{\pi_2},\ \ldots,\ \Lambda^{\pi_\mathcal{P}})     \\
    \text{s.t. }     & \Lambda^{\pi_1} \le \Lambda^{\pi_2} \le \ldots \le \Lambda^{\pi_\mathcal{P}} \\
  \end{aligned}
\end{equation}

\subsubsection*{Local Moves}

The local moves considered for this problem are as follows:

\begin{enumerate}
  \item Assigning a server to a pool and segment if there is an available one.
  \item Removing a server from a segment.
  \item Changing the segment to which a particular server is allocated, if other
        segments that can accommodate its size are available.
  \item Changing the pool to which a server is allocated, moving it to a different pool.
  \item Swapping the pools of two assigned servers, thereby transferring the
        capacity of the servers between pools.
  \item Swapping the segment in which servers are assigned. This is applicable
        when there is available space on both sides of the transfer to accommodate the
        change.
\end{enumerate}

Notably, these operations in the context of the modeling framework
implementation are encoded in the~\texttt{LocalMove} class.

\subsubsection*{Perturbation}

Regarding the perturbation of the solution, for this problem, we did not employ
any specific operation apart from randomly applying the described local moves.