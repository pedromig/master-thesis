In this section, we will provide a formal perspective on the developed problem
model, outlining the essential elements that define it. We will begin with the
problem instance, where we describe the data that characterizes a problem
instance. Next, we will delve into the concept of a solution within the context
of this problem. Finally, we will detail the combinatorial and neighborhood
structures which relate to~\acrshort{constructive-search} and~\acrshort{local-search} approaches.

\subsection{Problem}

As previously discussed in~\Cref{sec:odc-problem}, the problem revolves around
the placement of servers within data center rows and their allocation to
resource pools. This allocation is subject to space constraints imposed by the
number of slots available in each row and the potential unavailability of
certain slots. Therefore, to fully characterize a specific problem instance, the
following data must be known:

\begin{description}
  \item[\textbf{Servers.}] A description of the servers to be allocated.
    \begin{equation}
      \text{servers} = \{(s_m, c_m) \mid m = 0, 1, 2, \ldots, \mathcal{M}\}
    \end{equation}
    Where, $s_{m}$ and $c_{m}$ denote the size (in slots) and computing capacity of server $m$, respectively.
  \item[\textbf{Pools.}] A description of the available resource pools.
    \begin{equation}
      \text{pools} = \{c_{p} \mid p = 0, 1, 2, \ldots, \mathcal{P}\}
    \end{equation}
    Where, $c_{p}$ is the total capacity allocated to pool $p$.
  \item[\textbf{Rows.}] A description of the rows of the data center.
    \begin{equation}
      \begin{aligned}
        \text{rows} & = \{r_i \mid i = 1, 2, \ldots, \mathcal{R}\}                                  \\
        r_i         & = \{r_{i,1}, r_{i,2}, \ldots, r_{i,j} \mid  j = 0, 1, 2 \ldots, \mathcal{S}\}
      \end{aligned}
    \end{equation}
    Here, we introduce the binary variable $r_{i,j}$, which denotes whether, for
    a given row $i$, the slot $j$ is available (1) or unavailable (0).
  \item[\textbf{Unavailable Slots.}] A description of the unavailable slots for each row.
    \begin{equation}
      \text{unavailable} = \{ (i, j) \mid r_{i, j} = 0 \}
    \end{equation}
\end{description}

\subsection{Component}

% What is the ground set, what are components (pool, server, segment), Explain segments here
% Empty Solution 
% Partial Solution
% Complete Solution

% Data that characterizes a solution (pool, server, row)

\subsection{Solution}

A solution for this problem can then be described as an allocation of a server
to a specific row and a specific pool so long as that row can accommodate the server.
Formally, we can represent a solution as a set of allocations where each allocation is

\subsection{Objective Function}

% Talk about objective function
% Talk about surrogate model aka. (proxy) objective function for this problem?

\subsection{Combinatorial Structure}

\subsubsection*{Construction Rules}
% Talk about constructions (normal, sequence, heuristic)

\subsubsection*{Upper Bound}
% Explain all the bounds here 

\subsection{Neighborhood Structure}

% Explain Possible local moves