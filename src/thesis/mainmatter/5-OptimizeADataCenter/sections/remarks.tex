In this chapter, we provided our analysis and described model for the Google Hash
Code problem entitled~\textquote{\nameref{subsubsec:hashcode-2015-qualification}}.
Despite having only a single instance, which seemed simple, this problem posed a
several optimization challenges, mainly due to its bottleneck objective
function.

The task of optimizing the implementation of the model within the framework was
a meticulous process that required extensive thought and many hours of
development. It involved the incremental calculation of various values to
achieve computational efficiency, enabling solvers to address the problem within
a reasonable time frame. This work has provided valuable insights into the
effective modeling of problems and the design of robust upper bounds. While
progress has been made, there are still ideas that can be explored, ranging from
different component enumerations to more advanced bound techniques, which we
plan to explore the future.

The process of modeling posed significant challenges; however, it underscored
the effectiveness of this modeling approach. It offers a structured method for
problem-solving, which, when translated into implementation, enables rapid
testing with a variety of algorithms without the need for additional development
efforts, as the models can be reused. Furthermore, having these models in place
allows for a more comprehensive study of the performance
of~\acrshort{meta-heuristic} methods in the future. This will enable the
assessment of which algorithms work best for this problem.