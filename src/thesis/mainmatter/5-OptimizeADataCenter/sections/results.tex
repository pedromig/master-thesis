After formally describing the problem model, the information was translated into
a practical implementation developed within the modeling framework mentioned in
Chapter \ref{ch:principled-modelling-framework}. This implementation enabled the
use of various meta-heuristic methods for experimentation. Multiple iterations
of the model were tested, each involving different types of component
enumerations, bounds, heuristics, and various solvers.

The machine used to run the solvers on the single available instance for this
problem had the following s specifications:

\begin{table}[h]
  \centering
  \begin{tabular}{@{\extracolsep{4pt}}ccc@{\extracolsep{4pt}}}
    \toprule
    OS                 & CPU                                     & RAM   \\ \midrule
    Ubuntu 22.04.2 LTS & Intel i7-12700H (20 cores) --- 4.600GHz & 64 GB \\
    \bottomrule
  \end{tabular}
  \caption{Benchmark Machine Specifications}
\end{table}

The best results were achieved through a simple heuristic strategy in a
constructive search phase. This strategy involved enumerating ten first
components at each iteration using the heuristic strategy described
in~\Cref{algorithm:odc-enum-heuristic}. The selected component to add to the
solution was the one that contributed the most increment to the upper bound
value. Notably, the upper bound function used was the one described in Equation
\ref{eq:odc-upper-bound-2}, which effectively discriminated the results,
allowing for a construction that yielded a final score of 386 points in only 1
second.

Other~\acrshort{constructive-search} algorithms, such
as~\acrshort{grasp},~\acrshort{iterated-greedy} and~\acrshort{beam-search}
proved to be slower in constructing solutions of similar quality compared to the
simple heuristic strategy.~Multiple~\emph{add-hoc} parametrizations were tested
for each of these algorithms, but they consistently produced initial results
ranging from 280 to 320 points, taking additional time to further improve them
as to match the described heuristic.

For~\acrshort{local-search} strategy, the best meta-heuristic
was~\acrshort{iterated-local-search}. When run with a total timeout time of 15
seconds and using a perturbation with a kick strength of
2,~\acrshort{iterated-local-search} consistently improved the solution returned
by the heuristic construction strategy (regardless of the seed), yielding a
score of 410 points.~This result was satisfying and led to us move on to the
next problem.

Notably, the score of~\textbf{410} is better than the best score in the
competition leaderboard~(\textbf{407}).