This section introduces a potential implementation of solvers using the proposed
framework. The aim is to provide an overview of how meta-heuristic algorithms
can be developed utilizing the methods provided by the model API. To accomplish
this, we will start by describing a basic solver that utilizes heuristic
information for constructing solutions. Subsequently, we will demonstrate how
solvers for both~\acrshort{constructive-search} and~\acrshort{local-search}
methods can be implemented, using~\acrshort{beam-search}
and~\acrshort{iterated-local-search} meta-heuristics as illustrative examples.

\subsection{Heuristic Construction}
\label{subsec:heuristic-construction}

\Cref{lst:hc} describes a possible implementation of a
~\acrshort{meta-heuristic} solver that iteratively constructs a solution using
only heuristic information. This solver takes as input a~\texttt{Problem}
instance and returns a~\texttt{Solution}.

\lstinputlisting[float=ht,caption={Heuristic Construction Solver Implementation},label=lst:hc]{../assets/code/hc.py}

This solver can be summarized as follows:

\begin{enumerate}
      \item \textbf{Initialization}: Initialize a new empty~\texttt{Solution} and
            select a~\texttt{Component} to be added heuristically~(lines 2-3)
      \item \textbf{Construction}: Build the solution~(lines 4-6).
            \begin{enumerate}
                  \item If there is a component (\texttt{c}) available for addition to the
                        current solution,~(\ie{},~\texttt{c} is not the~\texttt{None}
                        sentinel value), proceed; otherwise stop the~\textbf{Construction}.~(line 4)
                  \item Add the selected component to the solution~(line 5).
                  \item Heuristically select a new component~(line 6).
                  \item Repeat the process.
            \end{enumerate}
      \item \textbf{Termination} Return the final~\texttt{Solution}~(line 7).
\end{enumerate}

In fact, this simple solver can serve as practical tool for testing the
implementation of a model. Its simplicity arises from its reliance on only two
fundamental methods: the~\texttt{add} method, which introduces a component to
the solution, and the~\texttt{heuristic\_add\_move} method, which determines the
next component to be added using a heuristic criterion. For testing purposes,
other uncomplicated~\acrshort{meta-heuristic} solvers can be crafted using the
same model. For instance, solvers that handle construction in a random or a
in greedy manner, based on components that yield the most favorable increments
with respect to the objective value and/or bound.

From an implementation perspective, this particular version takes a problem
instance as input and uses it to generate an initial empty solution.~However,
this method could have been designed to accept a partial solution and then
complete its construction from that point onwards.~Albeit, the option
to provide the problem instance rather than a solution as a parameter was chosen
primarily for illustrative purposes and to showcase the interaction between
the~\texttt{Problem} and the~\texttt{Solution} data structures.

\subsection{Beam Search}
\label{subsec:beam-search-solver}

\Cref{lst:bs} presents a possible implementation of a~\acrshort{beam-search}
constructive~\acrshort{meta-heuristic} solver. This~\acrshort{meta-heuristic},
as discussed earlier in~\Cref{ch:background}, can be viewed as having two phases. In the
first phase, it generates a candidate list of partial solutions that extend the
solutions present in a limited-size archive. Then, in the second phase, it
filters and selects the best solutions based on either the upper bound value or
a heuristic value. The selected solutions are retained in the archive for the
next iteration. Additionally, during this process, if a feasible solution is
encountered, the best solution found thus far is updated accordingly.

The execution time of this algorithm is not limited a stopping criterion is
required. In this implementation, we define the stopping condition as a time
budget, which is represented by the~\texttt{Timer} data structure.
The~\texttt{finished} method of the~\texttt{Timer} indicates whether the
allocated time budget has been used up. If the time budget is exhausted, the
algorithm stops.

Additionally, this solver takes other parameters as input in this
implementation.~These include a~\texttt{Problem} instance and the beam
width~(\texttt{bw}), which specifies the size of the solution archive. The
choice for supplying the problem as a parameter instead of a solution follows
the same motivation as previously mentioned in~\Cref{subsec:heuristic-construction}.

\lstinputlisting[float=ht,caption={Beam Search Solver Implementation},label=lst:bs]{../assets/code/bs.py}

In brief, this~\acrshort{beam-search} solver implementation is described by the
following steps:

\begin{enumerate}
      \item \textbf{Initialization}: Start by initializing an empty~\texttt{Solution}. If
            this initial solution is~\texttt{feasible}, store it as the best solution found so far.
            Add a tuple representing this solution to the archive~(\texttt{beam}), including its
            upper bound value.~(lines 2-5)
      \item \textbf{Construction}: Construct the solution while there is still available time in the budget~(lines 6-22).
            \begin{enumerate}
                  \item Create an empty candidate list~(\texttt{cs})~(line 7).
                  \item \textbf{Branch}: Expand the current solutions in the archive by adding
                        components to create new partial solutions~(lines 8-10).
                        \begin{enumerate}
                              \item Select a solution from the archive~(line 8).
                              \item Enumerate all possible components that can be added to that solution~(line 9).
                              \item Add these new solutions to the candidate list~\texttt{cs} as tuples containing
                                    the upper bound value, the partial solution, and the component leading to
                                    this contribution~(line 10).
                              \item Repeat while possible.
                        \end{enumerate}
                  \item If the candidate list~(\texttt{cs}) is not empty, filter the~\texttt{bw} best
                        candidate solutions based on their upper bound values.~(lines 11-13)
                  \item Empty the the~\texttt{beam} list~(line 14).
                  \item \textbf{Update}: Build the beam for the next iteration and update best solution~(lines 15-22).
                        \begin{enumerate}
                              \item Select a candidate solution (make a~\texttt{copy}) to
                                    potentially update the best solution~(lines 16).
                              \item Add the respective component to the current candidate solution~(line 17).
                              \item If the solution is feasible and its objective value is better than
                                    the best solution found so far, update the best solution by replacing
                                    it with the current one~(lines 18-21).
                              \item Add the candidate solution to the~\texttt{beam} list~(line 22).
                              \item Repeat while possible.
                        \end{enumerate}
            \end{enumerate}
      \item \textbf{Termination} Return the best~\texttt{Solution} found~(line 23).
\end{enumerate}

It is important to note that, this algorithm requires the implementation of
certain methods within the model, specifically in~\texttt{Solution} class.
However, the algorithm can still function successfully even if some of the
remaining methods are not implemented, as long as these mandatory methods are
provided.

\subsection{Iterated Local Search}
\label{subsec:ils}

The~\Cref{lst:ils} presents a possible implementation of
an~\acrshort{iterated-local-search} solver which is
a~\acrshort{local-search}~\acrshort{meta-heuristic}. As discussed
in~\Cref{ch:background}, This meta-heuristic commences with a feasible solution
and subsequently employs a~\acrshort{local-search} procedure.~Upon discovering a
solution that outperforms the current best solution, the algorithm updates the
best solution, and optionally perturbs the solution, before repeating the entire
process. In our implementation, we use~\acrshort{first-improvement} for
as the~\acrshort{local-search} strategy.

Similarly to the previous section's implementation of the~\acrshort{beam-search},
this algorithm is also not limited by execution time and requires a stopping
criterion. Here, we utilize a time budget represented by the~\texttt{Timer}
object, as previously discussed. Additionally, this solver accepts several other
input parameters. These include a~\texttt{Solution} instance and the
\textit{kick strength}~(\texttt{ks}), which determines the magnitude of the
perturbation to be applied to the solution.

\lstinputlisting[float=ht,caption={Iterated Local Search Solver Implementation},label=lst:ils]{../assets/code/ils.py}

In brief, this~\acrshort{iterated-local-search} solver implementation is
described by the following steps:

\begin{enumerate}
      \item \textbf{Initialization}: Begin with the current best solution as the starting solution.~(lines 2-3)
      \item \textbf{Local Search}: Conduct local search while there is available time in the budget.~(lines 4-21)
            \begin{enumerate}
                  \item \textbf{First Improvement}: Enhance the solution through the first local move that results in
                        a better objective value.~(lines 5-14)
                        \begin{enumerate}
                              \item Randomly pick a local move from the neighborhood of the current solution.~(line 5)
                              \item Calculate the incremental change in the objective value due to the local move.~(line 6)
                              \item Check if applying the current local move results in a better solution.~(line 7)
                              \item If improvement is achieved, apply the local move and proceed to the \textbf{Update} stage.~(lines 8-9)
                              \item If the time budget is exhausted during the first improvement stage, halt the \textbf{Local Search}
                                    and goto~\textbf{Termination}.~(lines 10-14)
                              \item Repeat this process until all possible local moves have been explored.
                        \end{enumerate}
                  \item \textbf{Update}: Update the current best solution if an improvement or a
                        different solution of the same quality was achieved during the first
                        improvement stage.~(lines 15-20)
                  \item \textbf{Perturb}: Optionally, perturb the current solution with a
                        kick strength of \texttt{ks}, if an improvement was obtained in the
                        first improvement stage.~(line 21)
            \end{enumerate}
      \item \textbf{Termination} Return the best~\texttt{Solution} found.~(lines 22-15)
\end{enumerate}

\subsection{Outline}
\label{subsec:solver-outline}

This section demonstrated the process of implementing solvers using the provided
modeling framework.~We provided practical examples for
both~\acrshort{constructive-search} and~\acrshort{local-search} meta-heuristics.
However, it is important to note that in this work, all the meta-heuristics
described in~\Cref{ch:background} were
implemented~\cite{rodriguesnasf4niopy}.~Furthermore, each of these
meta-heuristics requires different methods from the model, which we summarize
in~\Cref{tab:mh-api-methods}.

\begin{table}[h]
      \centering
      \renewcommand{\arraystretch}{1.2}
\begin{adjustbox}{max width=\textwidth}
  \begin{tabular}{@{}ll@{}}
    \toprule
    \textbf{Meta-Heuristic}                           & \textbf{Methods}                                                          \\
    \midrule
    \multirow{1}{*}{\acrshort{beam-search}}           & \begin{tabular}{ll}
                                                          \texttt{copy}                         & \texttt{objective}  \\
                                                          \texttt{upper\_bound}                 & \texttt{feasible}   \\
                                                          \texttt{add}                          & \texttt{add\_moves} \\
                                                          \texttt{upper\_bound\_increment\_add} &                     \\
                                                        \end{tabular}               \\
    \midrule
    \multirow{1}{*}{\acrshort{grasp}}                 & \begin{tabular}{ll}
                                                          \texttt{copy}                         & \texttt{objective}  \\
                                                          \texttt{add}                          & \texttt{feasible}   \\
                                                          \texttt{upper\_bound\_increment\_add} & \texttt{add\_moves} \\
                                                        \end{tabular}               \\
    \midrule
    \multirow{1}{*}{\acrshort{iterated-greedy}}       & \begin{tabular}{ll}
                                                          \texttt{copy}                         & \texttt{objective}            \\
                                                          \texttt{add}                          & \texttt{feasible}             \\
                                                          \texttt{upper\_bound}                 & \texttt{remove}               \\
                                                          \texttt{add\_moves}                   & \texttt{random\_remove\_move} \\
                                                          \texttt{upper\_bound\_increment\_add} &                               \\
                                                        \end{tabular}     \\

    \midrule
    \multirow{1}{*}{\acrshort{aco}}                   & \begin{tabular}{ll}
                                                          \texttt{copy}                         & \texttt{feasible}     \\
                                                          \texttt{objective}                    & \texttt{add}          \\
                                                          \texttt{add\_moves}                   & \texttt{upper\_bound} \\
                                                          \texttt{upper\_bound\_increment\_add} & \texttt{components}   \\
                                                        \end{tabular}             \\

    \midrule
    \multirow{1}{*}{\acrshort{first-improvement}}     & \begin{tabular}{ll}
                                                          \texttt{step}                      & \texttt{objective\_increment\_local} \\
                                                          \texttt{random\_local\_moves\_wor} &                                      \\
                                                        \end{tabular} \\
    \midrule
    \multirow{1}{*}{\acrshort{best-improvement}}      & \begin{tabular}{ll}
                                                          \texttt{step}                        & \texttt{local\_moves} \\
                                                          \texttt{objective\_increment\_local} &                       \\
                                                        \end{tabular}              \\
    \midrule
    \multirow{1}{*}{\acrshort{iterated-local-search}} & \begin{tabular}{ll}
                                                          \texttt{objective}                 & \texttt{copy}                        \\
                                                          \texttt{step}                      & \texttt{perturb}                     \\
                                                          \texttt{random\_local\_moves\_wor} & \texttt{objective\_increment\_local} \\
                                                        \end{tabular} \\
    \midrule
    \multirow{1}{*}{\acrshort{simulated-annealing}}   & \begin{tabular}{ll}
                                                          \texttt{copy}                        & \texttt{objective}                 \\
                                                          \texttt{step}                        & \texttt{random\_local\_moves\_wor} \\
                                                          \texttt{objective\_increment\_local} &                                    \\
                                                        \end{tabular} \\

    \midrule
    \multirow{1}{*}{\acrshort{tabu-search}}           & \begin{tabular}{ll}
                                                          \texttt{copy}                        & \texttt{objective}                 \\
                                                          \texttt{step}                        & \texttt{random\_local\_moves\_wor} \\
                                                          \texttt{objective\_increment\_local} & \
                                                        \end{tabular} \\
    \bottomrule
  \end{tabular}
\end{adjustbox}
      \caption{Required Methods from Model API for the Implemented Meta-Heuristics}
      \caption*{\small Note: Only the essential methods according to our
            implementation are presented, although other implementations may utilize
            different methods. Additionally, for the context of~\acrshort{aco}, our
            implementation adopts the~\emph{Max-Min Ant System} variant~\cite{stutzle1999maxmin}. Nevertheless, the
            method prerequisites should generalize to various other variants of this~\acrshort{meta-heuristic}.}
      \label{tab:mh-api-methods}
\end{table}