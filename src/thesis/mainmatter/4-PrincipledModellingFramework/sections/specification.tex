This section, establishes a collection of operations that a model must
support, all while motivating their necessity and applicability in the
context of meta-heuristic solver development. To achieve this, our description
will draw from the Python-based implementation that we've devised. However, it is
crucial to underline that our implementation serves as a demonstration of a
plausible way to materialize an~\acrshort{api} that facilitates modeling
operations. As such, considerations such as naming conventions and programming
language-specific concepts ought to be viewed as secondary.

Moreover, it is important to underscore that this framework represents a
consolidation of prior efforts undertaken by~\citet{vieira2009uma,
  fonseca2021nasf4nio, outeiro2021application}.~Consequently, it should be
perceived as an enhancement of the prerequisites and specifications established
in those works, tailored to encompass constructive and local search techniques,
as well as the comprehensive modeling procedure.

\subsection{Data Structures}

This framework is based around a set data structures that are standard and
utilized as a way to return information from the model. This design allows for
data hiding and isolation of the problem-specific details, which
any~\acrshort{meta-heuristic} solver can query and manipulate in a black-box
fashion. In the our implementation of the framework this corresponds to the
following four classes.

\begin{itemize}
  \item \inlinecode{class Problem: ...}\\
        Contains data that is known a priori and fully characterizes
        a problem instance. It is not meant to be modified by the solver.

  \item \inlinecode{class Solution: ...}\\
        Holds data that defines an empty or partially
        complete solution. It serves as the mutable state that a solver can
        modify during the optimization process.

  \item \inlinecode{class Component: ...}\\
        Characterizes any component within the ground set of a given problem.

  \item \inlinecode{class LocalMove: ...}\\
        Characterizes any local move that can be applied to a solution
\end{itemize}

Concerning the~\texttt{Problem} class, since it holds problem-specific data, it
might be advantageous to include methods within it for loading or parsing input
data, in addition to the class constructor. However, the primary intent remains
focused on storing problem-specific data and providing a away to initialize a
solution for a specific problem instance. This can be achieved through the
following method, which returns an empty solution.

\begin{center}
  \inlinecode{def empty_solution(self: Problem) -> Solution: ...}
\end{center}

It is important to acknowledge that the definition of an empty solution is tied
to the specific problem under consideration. Nevertheless, this method when
invoked returns an encapsulated solution object that abstracts the problem's details.

As previously mentioned, the~\texttt{Component} class serves as a representation
of a component from the ground set, encapsulating its relevant data.
Nonetheless, certain~\acrshort{meta-heuristic} methods, such as,~\acrshort{aco},
require a means for identifying specific components during solution
construction. To address this requirement, the following method was
incorporated:

\begin{center}
  \inlinecode{def id(self: Component) -> Hashable: ...}\\
\end{center}

As can be inferred by the methods's type signature, this method should return
an unique~(\texttt{Hashable}) identifier for a given component.

Concerning the~\texttt{LocalMove} class, it primarily encapsulates information
about a specific local move applicable to a particular solution, and it does not
necessitate any additional functionality.~Notably, these two classes are
frequently used in the model as they embody the dynamic aspects, encompassing
actions capable of altering solution states. Thus, from a practical standpoint,
their instances should be kept lightweight as they can have a significant impact
on performance.

The~\texttt{Solution} class is of extreme importance as it represents object
undergoing optimization. Apart from housing data fields that correspond to the
solution representation -- which will be populated during solver execution --
this class also contains the method definitions that we consider fundamental and
which describe the~\emph{model}. Consequently, in the subsequent sections, we
will comprehensively detail the methods within this class. These methods
represent the operations that a model should implement to
allow~\acrshort{constructive-search},~\acrshort{local-search} procedures and
other essential functionality.

\subsection{Inspection Methods}

The methods outlined in this section provide access to metrics that characterize
the state a solution. In terms of implementation, these function
as~\emph{getters} for the model's properties and should not be utilized for
executing resource-intensive computations, given the high frequency at which
they are meant to be invoked by solvers.

\subsubsection*{Objective Value}

\begin{center}
  \inlinecode{def objective(self: Solution) -> Optional[T]: ...}
\end{center}

This method returns the objective value of a solution. In the type signature for
this method, the return type~\texttt{T} signifies a generic type.~While this
type is usually numeric, it can be any other ~\emph{comparable} type.~This
requisite is fundamental from the solver's standpoint, as it's through comparing
this value that a solver evaluates whether a particular solution outperforms
others.

The wrapping~\texttt{Optional} type indicates that a method has the option to
either return a value or, if the value is not available or not defined, it can
emit a sentinel value. In our implementation, this sentinel value is represented
by~\texttt{None}.~This convention will be consistently applied to all methods
where the ~\texttt{Optional} type is employed.

\subsubsection*{Upper Bound}

\begin{center}
  \inlinecode{def upper_bound(self: Solution) -> Optional[T]: ...}
\end{center}

This method returns the upper bound value of a given solution.~Similar to the
\texttt{objective} method, the parameter~\texttt{T} in the function's type
signature signifies a generic type, typically numeric.

\subsubsection*{Feasibility}

\begin{center}
  \inlinecode{def feasible(self: Solution) -> bool: ...}
\end{center}

This method can be used to assess the feasibility of a candidate solution,
yielding a Boolean value of~\texttt{True} when the solution is feasible, and
~\texttt{False} otherwise.

\subsection{Constructive Search}

The subsequent methods delineate modeling operations
for~\acrshort{constructive-search}. As such, these methods revolve around
operations involving components. Specifically, these action involve enumeration,
selection, application, and evaluation of their contribution with respect to the
objective and upper bound values.

\subsubsection*{Component Enumeration}

From a~\acrshort{constructive-search} perspective, a central operation involves
recognizing the components that constitute the problem's ground set -- those
that can be added, or removed from a specific solution.~Therefore, it is
imperative for a model to offer a means through which a solver can access this
essential information. This is accomplished in our implementation via the
following set of methods.

\begin{center}
  \inlinecode{def add_moves(self: Solution) -> Iterable[Component]: ...}

  \inlinecode{def remove_moves(self: Solution) -> Iterable[Component]: ...}
\end{center}

Notably, all these methods yield a list of components specific to their
corresponding operations, denoted as~\texttt{Iterable[Component]} in the return
type signature. However, in our implementation, these methods are implemented
as~\emph{iterators}, signifying that upon invocation, they provide the first
available item and then pause until the next call to the same method is
initiated, essentially maintaining state. This approach holds significance from
a performance standpoint, both in conserving memory, as storing all components
for a given problem could be infeasible, and in terms of execution speed, since
there's no necessity to enumerate more components than what is required.

Furthermore, the intention behind these methods is for them to yield all
conceivable components that can be either added or removed from a specific
solution.~However, in practice, this might become unfeasible if the ground set
is excessively large. In such cases, an alternative strategy for component
enumeration may need to be contemplated.~It's important to note that this
consideration is relevant to the implementation of the method itself by the user
who models the problem, but it doesn't alter the method signature in any way.


Finally, within the context of~\acrshort{aco}, there's a need to enumerate all
components that are already included in a solution. This requirement is due to
the necessity to update the~\textit{pheromone model}. As a result, this
justifies the introduction of the following method.

\begin{center}
  \inlinecode{def components(self: Solution) -> Iterable[Component]: ...}
\end{center}

\subsubsection*{Heuristic Component Enumeration}

In the context of component enumeration, it can be advantageous to follow a
problem-specific heuristic order.~With this in mind, we introduced the
~\texttt{heuristic\_add\_moves} method, designed to offer an iterator for
components enumerated in accordance with the heuristic. Additionally, we
introduced the~\texttt{heuristic\_add\_move} method to provide a means for
exclusively returning a single heuristic move, if available.~The type signatures
for both methods are shown below.

\begin{center}
  \inlinecode{def heuristic_add_moves(self: Solution) -> Iterable[Component]: ...}

  \inlinecode{def heuristic_add_move(self: Solution) -> Optional[Component]: ...}
\end{center}

\subsubsection*{Random Component Selection}

As components can be enumerated in a heuristic order, there are situations where
selecting components randomly is advantageous. This is particularly relevant,
for instance, during processes like random solution construction or
destruction~(\acrshort{iterated-greedy}). The subsequent methods, when called,
provide randomized components that can potentially be added or removed,
contingent upon the current state of the solution.

\begin{center}
  \inlinecode{def random_add_move(self: Solution) -> Optional[Component]: ...}

  \inlinecode{def random_remove_move(self: Solution) -> Optional[Component]: ...}
\end{center}

\subsubsection*{Component Application}

It is essential for all constructive search methods to possess a mechanism for
incorporating or removing a component from a solution under construction.
Therefore, the ensuing methods have been introduced, and upon invocation, allow
the addition and removal of the specified component to/from the solution.

\begin{center}
  \inlinecode{def add(self: Solution, c: Component) -> None: ...}

  \inlinecode{def remove(self: Solution, c: Component) -> None: ...}
\end{center}

\subsubsection*{Objective Increment Calculation}

When evaluating the contribution of a component to the objective value of a
solution, there are two approaches to consider. The first approach involves
recalculating the entire objective value from scratch each time a component is
added or removed from the solution (full evaluation). The second approach
involves incremental evaluation, where only the change in objective value due to
the addition or removal of the component is computed. The latter method is
more efficient as it avoids the potentially expensive operation of recomputing
the entire objective value and should be preferred for its performance advantages.

This fact motivates the overall design of the~\texttt{Solution}~\acrshort{api}
and thus all operations involving evaluations are considered in function of
their increments. Albeit, it is possible for a user to implement
a full evaluation and return only the increment, but that aspect is
practical and beyond the scope of this specification.

That being said, the following methods are exposed, granting a solver the
capability to quantify the objective value contribution (increment) that is
obtained from adding or removing a designated component with respect to the
current state of the solution.

\begin{center}
  \inlinecode{def objective_increment_add(self: Solution, c: Component) -> Optional[T]: ...}

  \inlinecode{def objective_increment_remove(self: Solution, c: Component) -> Optional[T]: ...}
\end{center}

The return type of these methods is generic (denoted as~\texttt{T}), with the
precondition that it must be comparable and for all increment related methods it
must support the basic arithmetic operations of addition and subtraction.

\subsubsection*{Upper Bound Increment Calculation}

Just as there's a need to calculate increments concerning the objective value,
the same functionality is essential for upper bounds. This leads to the
introduction of the following methods, which serve the purpose of quantifying
the increment in the upper bound resulting from the addition or
removal of a component from a solution. Once more, the preference should lean
towards incremental evaluation for determining this value, as it's often even
more resource-intensive than calculating a contribution to the objective value.

\begin{center}
  \inlinecode{def upper_bound_increment_add(self: Solution, c: Component) -> Optional[T]: ...}

  \inlinecode{def upper_bound_increment_remove(self: Solution, c: Component) -> Optional[T]: ...}
\end{center}

\subsubsection*{Component Heuristic Value}

The concept of a heuristic value is associated with a component and offers a
means to rank the importance of components with respect to the current solution
state via a problem-specific heuristic.~The method's type signature is as
follows.

\begin{center}
  \inlinecode{def heuristic_value(self: Solution, c: Component) -> \ \ \ \ \ Optional[T]: ...}
\end{center}

Notably, this method was introduced to offer an alternative approach for the
~\acrshort{grasp} algorithm to construct a randomized solution.

\subsection{Local Search}

The subsequent methods define modeling operations for~\acrshort{local-search}
approaches.~These methods are centered on operations concerning local moves,
encompassing their enumeration, selection, application, and evaluation.

\subsubsection*{Local Move Enumeration}

In a manner similar to the approach adopted for~\acrshort{constructive-search},
the same principle holds when addressing~\acrshort{local-search}. Here, the
necessity arises to enumerate all plausible alternatives for optimizing
(exploiting) a candidate solution, which in turn pertains to the exploration of
neighboring solutions. To fulfill this requirement, the following method serves
as a mechanism for enumerating all the local moves that can be executed on a
given solution. This method adheres to the same pattern as previous enumeration
methods as it is meant to function as an~\emph{iterator} for feasible local
moves.

\begin{center}
  \inlinecode{def local_moves(self: Solution) -> Iterable[LocalMove]: ...}
\end{center}

\subsubsection*{Random Local Move Enumeration}

While local moves can indeed be enumerated in a specific order, as enabled by
the~\texttt{local\_moves} method, the majority of local search strategies depend on random
processes for selecting moves. This motivation gives rise to the subsequent two
methods: one for systematically~\emph{iterating} all potential local moves in a
randomized manner without repeating the same local move twice,~\ie{}, without
replacement~(\texttt{random\_local\_moves\_wor} method), and the other for returning a random local
move~(\texttt{random\_local\_move} method).

\begin{center}
  \inlinecode{def random_local_moves_wor(self: Solution) ->\ \ \ \ \ \ \ \ \ \ \ \ \ \ Iterable[LocalMove]: ...}

  \inlinecode{def random_local_move(self: Solution) -> Optional[LocalMove]: ...}
\end{center}

When it comes to random enumeration of a candidate solution's neighborhood
without replacement, as performed by the ~\texttt{random\_local\_moves\_wor} method,
important insights can be obtained regarding its efficient implementation.

Conventionally, a naive method would involve shuffling all $M$ possible local
moves, selecting a local move, utilizing it, and then repeating the shuffling
process minus the $k$ moves that were rendered infeasible due to the use of the
chosen move. This would leave the remaining $M - k$ moves for selection.
Nonetheless, repeating this until exhausting all local moves would result in an
additional space complexity of $\mathcal{O}(M)$. However, as this approach can
be memory-intensive, we propose an alternative method that achieves the same
outcome with a space complexity of $\mathcal{O}(1)$.

The alternative approach entails employing a~\acrfull{lcg} to achieve the same
outcome with a constant spacial complexity, thereby eliminating the need for
additional space.

\begin{definition}[Linear Congruential Generator~\cite{knuth2014art}]
  A linear congruential generator is a pseudo-random number generator
  described by the following recurrence relation:

  \begin{equation}
    \mathcal{X}_{n+1} = (a\mathcal{X}_{n} + c) \bmod m
  \end{equation}
  Where,
  \begin{align*}
    m,               & \quad 0 < m \text{ — is the modulus}                                                             \\
    a,               & \quad 0 < a < m \text{ — is the multiplier}                                                      \\
    c,               & \quad 0 \leq c < m \text{ — is the increment}                                                    \\
    \mathcal{X}_{0}, & \quad 0 \leq \mathcal{X}_{0} < m \text{ — is the \textquote{seed} or \textquote{starting value}}
  \end{align*}
\end{definition}

For a~\acrshort{lcg} it is known~\cite{knuth2014art} there is a choice of
parameters that guarantees a maximal period for generator, and equal to $m$.
This in means that all the numbers until $m$ will be emitted by the generator in
a random order exactly once.Albeit, the choice of parameters must obey the
following conditions~\cite{knuth2014art}:

\begin{enumerate}
  \item $c$ is relatively prime to $m$
  \item $a - 1$ is a multiple of all prime factors dividing $m$
  \item $a - 1$ is multiple of 4, if m is a multiple of 4
\end{enumerate}

For example, the choice of the parameters $a = 5$, $m = 8$ and $c = 1$, yields
a~\acrshort{lcg} that has maximal period. Remarkably, the choice of $m$ as a
power of two has important implications in terms of performance from a
computational perspective, allowing for faster modulus operations.

With this in mind, a properly configured~\acrshort{lcg} can be harnessed for the
purpose of random local move enumeration without replacement. This is achieved
by associating each local move with a number $0 < i < m$, which, upon
generation, is decoded into the corresponding~\texttt{LocalMove} object.
Notably, the parameter $m$ needs to be adjusted in order to contemplate all
possible moves.

\subsubsection*{Local Move Application}

This method is designed to implement a problem-specific perturbation to a given
solution during~\acrshort{local-search}. The integer parameter in the method
signature provided below is employed to determine the~\textquote{kick strength},
which signifies the magnitude of the perturbation to be applied.

\begin{center}
  \inlinecode{def step(self: Solution, m: LocalMove) -> None: ...}
\end{center}

\subsubsection*{Local Move Objective Increment Calculation}

Similar to the approach used for~\acrshort{constructive-search}, where the
~\texttt{objective\_increment\_add} method is employed to evaluate the
incremental contribution of a particular component to the objective value, a
comparable functionality is necessary for local search. However, in this case,
the evaluation pertains to local moves. This requirement is addressed through
the following method.

\begin{center}
  \inlinecode{def objective_increment_local(self: Solution, m: LocalMove) -> Optional[T]: ...}
\end{center}

From an implementation perspective, similar to all incremental evaluations
(including those for~\acrshort{constructive-search}), this operation should
maintain efficiency.~Arguably, this could be one of the most frequently used
operations within an optimization process, as it enables the assessment of
whether a candidate solution has improved or worsened following a particular
action.

\subsubsection*{Perturbation}

This method serves the purpose of applying a problem-specific perturbation to
a given solution undergoing~\acrshort{local-search}.~The integer parameter in the
method signature shown bellow is used to identify the~\textquote{kick strength},~\ie{}
the intensity of the perturbation to be applied.

\begin{center}
  \inlinecode{def perturb(self: Solution, ks: int) -> None: ...}
\end{center}

\subsection{Utility Methods}

From an implementation perspective, there exists a need for a~\emph{copy} operation to
manage the preservation of the best solution(s) found in memory. This operation
prevents the unintended overwriting of a given solution due to modifications
made during the optimization process. In the specific context of this framework,
this operation applies to the~\texttt{Solution} object and is significant for both
\acrshort{constructive-search} and~\acrshort{local-search} approaches.

Although it might be perceived as an implementation detail, we consider that it
deserves its distinct place within the model definition. This stems from the
critical aspect that a solution object must consistently access the problem
object, which contains problem-specific information. However, during the copying
process of a solution, duplicating this information is usually unnecessary, and
referencing it suffices. With this in mind, the following method was, introduced
to address this requirement.

\begin{center}
  \inlinecode{def copy(self: Solution) -> Solution: ...}
\end{center}

\subsection{Outline}

In summary, this specification encompasses what we consider as the model within
the realm of meta-heuristics. While not formulated mathematically but rather in
computational terms, this representation through the described constructs, from
our perspective, effectively captures the attributes of a problem. It enables
the problem to be addressed in a black-box manner by solvers, which make use of
this standardized set of methods, as we will describe next,
in~\Cref{sec:solver}.~The modeling endeavor then involves implementing each of
these methods in accordance with their context within a particular problem.

The~\Cref{tab:api} serves as a summary of all the methods related to the model
~\acrshort{api} which were documented in this section and are part of our
implementation of the principled modeling framework~\cite{rodriguesnasf4niopy}.

\begin{table}[h]
  \centering
  \renewcommand{\arraystretch}{1.5}
\begin{adjustbox}{max width=\textwidth}
  \begin{tabular}{@{}llcp{4cm}cp{4cm}@{}}
    \toprule
    \textbf{Class}                      & \textbf{Methods}                         & \textbf{Input}     & \textbf{Output}              \\
    \midrule
    \multirow{1}{*}{\texttt{Problem}}   & \texttt{empty\_solution}                 & ---                & \texttt{Solution}            \\
    \midrule
    \multirow{23}{*}{\texttt{Solution}} & \texttt{copy}                            & \texttt{Solution}  & \texttt{Solution}            \\
                                        & \texttt{feasible}                        & ---                & \texttt{bool}                \\
                                        & \texttt{objective}                       & ---                & \texttt{Optional[T]}         \\
                                        & \texttt{upper\_bound}                    & ---                & \texttt{Optional[T]}         \\
                                        & \texttt{components}                      & ---                & \texttt{Iterable[Component]} \\
                                        & \texttt{add\_moves}                      & ---                & \texttt{Iterable[Component]} \\
                                        & \texttt{remove\_moves}                   & ---                & \texttt{Iterable[Component]} \\
                                        & \texttt{heuristic\_add\_moves}           & ---                & \texttt{Iterable[Component]} \\
                                        & \texttt{heuristic\_add\_move}            & ---                & \texttt{Optional[Component]} \\
                                        & \texttt{random\_add\_move}               & ---                & \texttt{Optional[Component]} \\
                                        & \texttt{random\_remove\_move}            & ---                & \texttt{Optional[Component]} \\
                                        & \texttt{local\_moves}                    & ---                & \texttt{Iterable[LocalMove]} \\
                                        & \texttt{random\_local\_moves\_wor}       & ---                & \texttt{Iterable[LocalMove]} \\
                                        & \texttt{random\_local\_move}             & ---                & \texttt{Optional[LocalMove]} \\
                                        & \texttt{add}                             & \texttt{Component} & \texttt{None}                \\
                                        & \texttt{remove}                          & \texttt{Component} & \texttt{None}                \\
                                        & \texttt{step}                            & \texttt{LocalMove} & \texttt{None}                \\
                                        & \texttt{perturb}                         & \texttt{int}       & \texttt{None}                \\
                                        & \texttt{heuristic\_value}                & \texttt{Component} & \texttt{Optional[T]}         \\
                                        & \texttt{objective\_increment\_local}     & \texttt{LocalMove} & \texttt{Optional[T]}         \\
                                        & \texttt{objective\_increment\_add}       & \texttt{Component} & \texttt{Optional[T]}         \\
                                        & \texttt{objective\_increment\_remove}    & \texttt{Component} & \texttt{Optional[T]}         \\
                                        & \texttt{upper\_bound\_increment\_add}    & \texttt{Component} & \texttt{Optional[T]}         \\
                                        & \texttt{upper\_bound\_increment\_remove} & \texttt{Component} & \texttt{Optional[T]}         \\
    \midrule
    \multirow{1}{*}{\texttt{Component}} & \texttt{id}                              & ---                & \texttt{Hashable}            \\
    \midrule
    \multirow{1}{*}{\texttt{LocalMove}} & ---                                      & ---                & ---                          \\
    \bottomrule
  \end{tabular}
\end{adjustbox}
  \caption{Modeling API Specification}
  \caption*{\small Note: The parameter~\texttt{T} denotes a generic comparable
    type.~In the case of the \textquote{increment} functions this type must also support
    addition and subtraction.~The class~\texttt{LocalMove} does not expose any methods,
    and was only added for completeness.
  }
  \label{tab:api}
\end{table}