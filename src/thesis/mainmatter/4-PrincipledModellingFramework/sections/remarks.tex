In this chapter, we a proposed framework for problem modeling that enables the
development of solvers capable of addressing problems in a black-box manner.
While we used our \emph{Python} implementation to illustrate the framework from
a modeling, solver development, and testing perspective, the fundamental
constructs of this framework can be extrapolated and applied to any programming
language. Our intent was to demonstrate the general principles that underlie the
framework. We applied this implementation of the framework to model two specific
problems in the Google Hash Code competition, which we will discuss
in~\Cref{ch:optimize-data-center,ch:book-scanning}.

The selection of~\emph{Python} as the programming language, for developing the
framework and models, was primarily driven by its ease of use, rapid prototyping
capabilities, and rich high-level features. However, the language interpreted
nature and relative slowness compared to compiled languages can impact the
efficiency of solver execution.~To mitigate this, we turned to an alternative
\emph{Python} implementation known as~\emph{PyPy}\footnote{\url{https://www.pypy.org/}}.~This just-in-time (JIT) compiled
version of Python has a substantial speed boost, claiming to be approximately
4.8 times faster than the implementation found and average interpreter~(\emph{CPython}).

In fact, this became particularly evident when dealing with complex problem
instances, where the performance difference was remarkable. While the
current~\emph{Python} implementation served its purpose for prototyping and
exploring ideas for problem-solving, a reimplemented version of this framework
in a more performant programming language would be necessary for optimal
performance. Nonetheless, rapidly testing new ideas from a modeling standpoint
this framework remains a valuable choice.

Lastly, it is important to acknowledge that the specification of the framework
might not be exhaustive. The range of~\acrshort{meta-heuristic} methods, as
documented in the literature~\cite{osman1996metaheuristics}, is extensive, and
new methods could potentially emerge in the future. While the range of
meta-heuristics covered in the framework is substantial, it is possible that
certain use-cases could necessitate the addition of new functionalities.
Therefore, the framework's completeness and stability will certainly depend on
insights from the community.