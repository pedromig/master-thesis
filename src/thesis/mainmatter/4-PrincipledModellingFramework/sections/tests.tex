As discussed, this framework offers capabilities in both problem modeling and
the implementation of meta-heuristic solvers. This versatility allows it to
satisfy the needs of two distinct user profiles: those primarily focused on
problem-solving, seeking access to a variety of solvers for effective results,
and~\acrshort{meta-heuristic} developers who concentrate
on~\acrshort{meta-heuristic} refinement.~The latter group employs the
framework's provided problems as a means to rigorously evaluate and test their
developed meta-heuristics. Albeit, there can also be individuals interested in both
aspects.

A common thread among all users of the framework is the need to test
implementations. From the vantage point of a~\acrshort{meta-heuristic}
developer, the primary objective revolves around constructing tests that assess
the meta-heuristic's correct implementation,~\eg{},~\textit{unit-tests}. From
the perspective of a model developer, these tests, while valuable for assessing
the correctness of the model's implementation, extend beyond mere verification
of whether methods yield the expected result. Instead, the tests should
encompass the evaluation of whether these outcomes adhere to fundamental
properties that should universally held true for any given input-output pair,
irrespective of the specific problem at hand. This type of testing, is known
as~\textit{property testing}.

To illustrate, let's take a scenario where a component we intent to add a
component to a solution.~In a conventional~\textit{unit-test} approach, one
could verify if the objective value of the solution matches the expected outcome
after adding that particular component. Yet, a broader property to consider is
that, irrespective of the specific objective value, the solution's value
following the component addition should be equivalent to the value prior to the
addition, plus the incremental contribution attributable to that specific
component. This property testing goes beyond specific cases and encapsulates the
core behavior of the operation.

\lstinputlisting[float=ht,caption={Objective Increment Add Property Test},label=lst:test]{../assets/code/test.py}

Importantly, the design of the modeling framework offers a user-friendly
interface for creating these tests, as exemplified in~\Cref{lst:test}. In
essence, the implementation of a property test for an objective value increment
is follows the subsequent steps:

\begin{enumerate}
      \item \textbf{Initialization}: Start by initializing a random solution and selecting
            a random component to add to the solution.~(lines 2-3)
      \item \textbf{Property Test}: Execute a property test by constructing a random solution.~(lines 4-9)
            \begin{enumerate}
                  \item If a random component (\texttt{c}) is available for addition to the current
                        solution~(\ie{},~\texttt{c} is not the~\texttt{None} sentinel value),
                        proceed; otherwise, conclude the~\textbf{Property Test}.~(line 4)
                  \item Record the objective value of the current solution and calculate the increment
                        achieved by applying the chosen component.~(lines 5-6)
                  \item Add the component to the solution.~(line 7)
                  \item Assert that the sum of the objective value before adding the component and
                        the calculated increment equals the current objective value. If this assertion
                        fails, the property is not verified.~(line 8)
                  \item Choose a new component and repeat the process.~(line 9)
            \end{enumerate}
      \item \textbf{Termination} Conclude the property test and return the~\texttt{Solution}.~(line 10)
\end{enumerate}

It is worth noting that this type of testing is inherently randomized, and it
should undergo multiple iterations to ensure statistical validation of the
implemented model's adherence to the properties. Furthermore, this test is
designed to take the~\texttt{Problem} instance as an input, illustrating the
potential for seamlessly integrating~(or~\emph{injecting}) these tests into
any~\texttt{Problem}.~This offers the same possibility as with implementing solvers ---
allowing others to also develop and contribute with test implementations in a
similar manner.

Additionally, this specific example of a property test is just a single
instance, with the potential for similar tests to be conducted for various types
of operations. For instance, one could examine the enumeration of local moves ---
whether performed in a random manner~(\texttt{random\_local\_moves\_wor}) or
following a specified order~(\texttt{local\_moves}) --- to validate that they
consistently yield identical components and maintain a consistent count of
components with each iteration.~This underscores the versatility and ease-of-use
of property tests in evaluating a range of aspects within the proposed framework.