\chapter{Conclusion}
\label{ch:conclusion}

\chquote{\textquote{We can only see a short distance ahead, but we can see plenty there that needs to be done.}}{Alan Turing}{}

In this thesis, our main objective was to address two complex combinatorial
optimization problems from the Google Hash Code competition using~\acrshort{meta-heuristic}
approaches.~Our approach was centered on developing conceptual models for these
problems and implementing them within a framework we designed. This enabled the
utilization of generic~\acrshort{meta-heuristic} solvers that were also developed to
conform to the framework's specification. Essentially, the framework acted as a
practical tool to bridge the gap between conceptual problem modeling and actual
problem-solving, while offering a platform for creating generic~\acrshort{meta-heuristic}
solvers that can tackle any problem in a black-box fashion.

Our efforts resulted in successful problem modeling and promising outcomes for
both of the problems we attempted to solve. Notably, all the problem models we
devised could be implemented through the framework, highlighting the
adaptability and effectiveness of our approach in tackling intricate problems.
Moreover, our framework allowed the development of all the~\acrshort{meta-heuristic}
strategies discussed in~\Cref{ch:background}, demonstrating that is indeed
possible to this. However, it's important to acknowledge that the framework we
constructed will benefit from further refinement, especially in terms of
accommodating different modeling aspects, such as describing models for other
types of meta-heuristics like evolutionary algorithms, which were not within the
scope of this work.

Overall, the experience of working with this framework was positive, primarily
due to its transparency in addressing essential modeling questions and its clear
separation of solvers and models, making a wide range of utilities readily
available. However, becoming proficient in the framework requires
training and conceptual familiarity, which will only come with practice.

\section{Future Work}

Possible research directions in future work are presented as follows:

\begin{description}
  \item[Problem Models] Expanding the repertoire of problem models, both for
    the existing Google Hash Code problems and for other challenges within the
    competition, holds significant importance. This broader range of models will
    enable a comprehensive assessment of how the framework performs when applied
    to problems with diverse characteristics. The models we've constructed so far
    represent just a small subset, and a more extensive selection will facilitate
    a more in-depth analysis of~\acrshort{meta-heuristic} behavior. This expansion is
    crucial, as having a variety of performant models will support extensive
    testing and refinement of solvers.

  \item[Meta-Heuristic Implementation] Further implementations of other
    meta-heuristic techniques, such as evolutionary algorithms, can be
    explored. The introduction of new meta-heuristics may necessitate updates
    and additional functionalities for the framework to accommodate these approaches effectively.

  \item[Experimental Evaluation of Meta-Heuristics] The results obtained from the
    developed models have not yet delved into the parametrization of meta-heuristic
    methods. This analysis is crucial for optimizing and fine-tuning these
    techniques to achieve the best possible performance in solving complex problems.
    Furthermore, future work should include a thorough examination and comparison
    of different  meta-heuristics.
\end{description}