The main contributions of this thesis related to the aforementioned research
questions, are as follows:

\vspace{0.5cm}

\begin{description}

  \item[\textbf{C1.}] With the existing research on principled modeling for
    meta-heuristics~\cite{vieira2009uma,fonseca2021nasf4nio,outeiro2021application},
    this document aims to consolidate and formalize a comprehensive specification.
    Our objective is to encapsulate all the concepts and developments made thus
    far. Additionally, we have created a practical Python implementation of this
    framework. In essence, both in the formalization and implementation, we
    endeavor to synthesize the existing ideas concerning modeling for constructive
    and local search techniques. These techniques are integral components of
    meta-heuristic solvers and from our perspective, they should be integrated
    into a model, as we will elaborate upon in this thesis.

  \item[\textbf{C2.}] We developed several meta-heuristic solvers and utilities
    both for gathering the solutions and for testing the developed models. Given
    that these are general-purpose they can work with any model that is developed
    under the practical implementation of the framework we devised.

  \item[\textbf{C3.}] After a careful selection process we selected two Google Hash Code
    problems for which some models for each of the problems were developed that
    explore the different properties of the problems in an attempt to both obtain the best solutions
    possible. Those models not only provide a practical example on how to model a problem but
    also explore the framework capabilities in great detail.
\end{description}