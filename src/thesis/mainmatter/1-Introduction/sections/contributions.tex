The main contributions of this thesis are related to the aforementioned research
questions, as follows:

\begin{description}

  \item[\textbf{C1.}] Building upon existing research on well-structured
    modeling for meta-heuristics~\cite{vieira2009uma,fonseca2021nasf4nio,outeiro2021application},
    this document aims to gather and formalize a comprehensive specification. Our
    goal is to bring together all the concepts and developments made so far. For
    this purpose, we have created a practical Python implementation of the
    framework, with the aim of summarizing the existing ideas related to modeling
    for both constructive and local search methods.

  \item[\textbf{C2.}] We implemented several meta-heuristic solvers and utilities
    both for gathering the solutions and for testing the developed models. Given
    that these are general-purpose tools they can work with any model that is developed
    under the practical implementation of the framework we devised.

  \item[\textbf{C3.}] We selected two Google Hash Code problems for which some
    models were developed that explore the different properties of each of the problems
    in an attempt to obtain the best possible solutions. Furthermore, these
    models provide a practical example on how to model relatively complex
    problems and also allow us to think critically about the framework capabilities.
\end{description}