The main contributions of this thesis related to the aforementioned research
questions, are as follows:

\begin{description}

  \item[\textbf{C1.}] With the existing research on principled modelling for
    meta-heuristics~\cite{vieira2009uma,fonseca2021nasf4nio,outeiro2021application},
    this document aims to consolidate and formalize a comprehensive specification.
    Our objective is to encapsulate all the concepts and developments made thus
    far. Additionally, we have created a practical Python implementation of this
    framework. In essence, both in the formalization and implementation, we
    endeavour to synthesize the existing ideas concerning modelling for
    constructive and local search techniques.

  \item[\textbf{C2.}] We implemented several meta-heuristic solvers and utilities
    both for gathering the solutions and for testing the developed models. Given
    that these are general-purpose they can work with any model that is developed
    under the practical implementation of the framework we devised.

  \item[\textbf{C3.}] We selected two Google Hash Code problems for which some
    models for each of the problems were developed that explore the different
    properties of the problems in an attempt to both obtain the best solutions
    possible. These models provide a practical example on how to model relatively
    complex problems and also allows us to think critically about the framework
    capabilities.
\end{description}