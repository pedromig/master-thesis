The main goal of this work is the implementation and evaluation of meta-
heuristic solution approaches for two Google Hash Code problems, using a
principled approach that separates the modeling of the problems from the
solvers.

In particular, we aim to expand upon the modeling approach for meta-heuristics
that has been partially explored in previous
research~\cite{vieira2009uma,fonseca2021nasf4nio,outeiro2021application}. The
objective is to solidify existing concepts while introducing additional
functionality, both in conceptual understanding and practical application.

Furthermore, we aim to construct models of Google Hash Code problems. These
models will not only be described and discussed in this thesis but will also
serve as illustrative examples documenting the modelling concepts. Furthermore,
they will enable a critical evaluation of the merits and shortcomings of this
principled approach in comparison to more ad-hoc and traditional methods of
problem-solving.

Finally, the implementation of state-of-the-art meta-heuristic solvers is a
vital component of our work as it will enable us to assess the performance and
quality of solutions found for the models of the Google Hash Code problems as
well as the feasibility of the modelling approach for meta-heuristic solver
development.

In summary, the main research questions we outline for this thesis are:

\begin{description}
  \item[\textbf{R1.}] Can existing ideas explored by previous work on modelling
    frameworks \cite{vieira2009uma,fonseca2021nasf4nio,outeiro2021application}
    be formalized and a practical implementation be developed, potentially contributing with new
    features?

  \item[\textbf{R2.}] Can general-purpose meta-heuristic solvers with respect to
    the principled modelling framework implementation?

  \item[\textbf{R3.}] Can Google Hash Code problems be solved effectively using
    this modelling approach?
\end{description}