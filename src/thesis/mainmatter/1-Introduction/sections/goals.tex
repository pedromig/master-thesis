This thesis focuses on creating, implementing, and evaluating meta-heuristic
approaches using a principled modelling framework. The Google HashCode Problems
serve as benchmarks for our study.

Our approach encompasses two primary aspects. First, from a modeling
perspective, we aim to expand upon the modelling concepts that have been
partially explored in previous
research~\cite{vieira2009uma,fonseca2021nasf4nio,outeiro2021application}. Our
primary objective is to solidify existing concepts while introducing additional
functionality, both in conceptual understanding and practical application.

Additionally, we aim to construct models for the Hash Code problems. These
models will not only be analyzed in this thesis but will also serve as
illustrative examples for explaining and teaching the modeling concepts.
Furthermore, they will enable a critical evaluation of the merits and
shortcomings of this structured approach in comparison to more ad-hoc and
traditional methods of problem-solving.

Secondly, the implementation of state-of-the-art meta-heuristic solvers. This
development enables a thorough examination of the models and the overarching
strategy. This assessment will encompass an analysis of both performance metrics
and the quality of solutions achieved through this approach.

As such, the main research questions we outline for this thesis are:

\vspace{0.5cm}

\begin{description}
  \item[\textbf{R1.}] Can we formalize the existing ideas explored by previous
    work on the modelling framework and produce a practical implementation,
    potentially contributing with new features?

  \item[\textbf{R2.}] Can we develop general-purpose meta-heuristic solvers and
    utilities with respect to the principled modelling framework implementation?

  \item[\textbf{R3.}] Can Google Hash Code problems be solved using a modelling approach?
\end{description}