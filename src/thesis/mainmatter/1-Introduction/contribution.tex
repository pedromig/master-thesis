\section{Contribution}
\label{section:contribution}

The main goal of this work is to develop and implement effective heuristic and
meta-heuristic approaches for solving Hash Code problems, with a particular
focus on the modelling aspect and the clear separation between solvers and
models. By doing so, we hope to develop more structured and efficient
problem-solving strategies that can effectively address a range of challenges.
Additionally, some effort will be made to address other key areas that are
crucial to the success of this work, namely:

\begin{itemize}
  \item Development and refinement of the frameworks that separate models from solvers
        currently materialized in an Application Programming Interface (API)
        designed only for constructive search~\cite{outeiro2021application}.
        The main goal is to optimize and ``fine-tune'' this API in order to improve its
        efficacy and utility, by using the Hash Code problems as benchmarks.

  \item Expansion of the aforementioned API to support local search strategies
        in a problem-independent manner. This will allow for its application to a wider
        range of problems and contexts.

  \item Implementation of a small set of general-purpose meta-heuristic solvers and utilities
        that can be used to not only generate and test solutions for the various
        Hash Code benchmark problems, but also to verify the correctness of the results.
\end{itemize}

Last but not least, the objective of this work is to engage in a critical
examination of the strengths and limitations of our adopted approach to problem
modelling and solver development. This discussion will be relevant to
meta-heuristic researchers, software developers, and practitioners alike. The
analysis will consider various performance dimensions, including the effort
required for problem modelling and solver development, the computational
efficiency of the implemented software, and the quality of the solutions
obtained.